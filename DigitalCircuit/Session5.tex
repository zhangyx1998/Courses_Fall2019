\section{Session 5 - Sequential circuit and unit}
\vspace{-15pt}\noindent\rule{\textwidth}{0.1pt}\vspace{-10pt}
    \subsection{Notes}
    {\color{hwSolution}
    \subsection*{Another method to generate delayed paluse - without losing its length}
        \begin{center}\begin{tikzpicture}
% CHIP 2 (Y+0.0)
    \draw[black,thick]
      (-1.2,2.4)--(-1.2,0.0)
    --( 1.2,0.0)--( 1.2,2.4)
    --cycle;
    % Left Side
    \draw[black,thick, -] (-1.2,1.6) node[right]{$A$}
    -- (-2.0,1.6) node[left]{$V_{DD}$};
    \draw[black,thick,o-] (-1.2, 0.8) node[right]{$B$}
    -- (-2.0,0.8) node[left]{\textit{GND}};
    % Right Side
    \draw[black,thick, -] ( 1.2, 1.6) node[left]{$Q$}
    -- ( 2.0, 1.6);
    \draw[black,thick,o-] ( 1.2, 0.8) node[left]{$\bar{Q}$}
    -- ( 1.4, 0.8);
    % Top Side
    \draw[black,thick, -] (-0.4, 2.4) node[below]{$C_X$}
    -- (-0.4, 3.0);
    \draw[black,thick, -] ( 0.4, 2.4) node[below]{$R_X$}
    -- ( 0.4, 3.0);
    % Bottom Side
    \draw[black,thick,o-*] ( 0.0, 0.0) node[above]{$C_{L}$}
    -- ( 0.0,-0.6) -- ( 1.8,-0.6);
    % C/R
    \draw[black,thick, -]
        % Capacitor
        (-0.4, 3.0) -- (-0.1, 3.0)
        ( 0.4, 3.0) -- ( 0.1, 3.0)
        ( 0.1, 3.2) -- ( 0.1, 2.8)
        (-0.1, 3.2) -- (-0.1, 2.8)
        ( 0.0, 3.2) node[above]{$C$}
        % Resistor
        ( 0.4, 3.0) -- ( 0.7, 3.0)
        ( 1.2, 3.0) -- ( 1.5, 3.0)
        ( 0.7, 3.1) -- ( 1.2, 3.1) -- ( 1.2, 2.9) -- ( 0.7, 2.9) -- cycle
        ( 0.95, 3.2) node[above]{$R$};
    \draw[black,thick,-*] ( 1.5, 3.0) -- ( 1.8, 3.0);
% CHIP 1 (Y+4.0)
    \draw[black,thick]
    (-1.2,2.4)--(-1.2,0.0)
  --( 1.2,0.0)--( 1.2,2.4)
  --cycle;
  % Left Side
  \draw[black,thick, -] (-1.2,1.6) node[right]{$A$}
  -- (-2.0,1.6) node[left]{$V_{DD}$};
  \draw[black,thick,o-] (-1.2, 0.8) node[right]{$B$}
  -- (-2.0,0.8) node[left]{\textit{GND}};
  % Right Side
  \draw[black,thick, -] ( 1.2, 1.6) node[left]{$Q$}
  -- ( 2.0, 1.6);
  \draw[black,thick,o-] ( 1.2, 0.8) node[left]{$\bar{Q}$}
  -- ( 1.4, 0.8);
  % Top Side
  \draw[black,thick, -] (-0.4, 2.4) node[below]{$C_X$}
  -- (-0.4, 3.0);
  \draw[black,thick, -] ( 0.4, 2.4) node[below]{$R_X$}
  -- ( 0.4, 3.0);
  % Bottom Side
  \draw[black,thick,o-*] ( 0.0, 0.0) node[above]{$C_{L}$}
  -- ( 0.0,-0.6) -- ( 1.8,-0.6);
  % C/R
  \draw[black,thick, -]
      % Capacitor
      (-0.4, 3.0) -- (-0.1, 3.0)
      ( 0.4, 3.0) -- ( 0.1, 3.0)
      ( 0.1, 3.2) -- ( 0.1, 2.8)
      (-0.1, 3.2) -- (-0.1, 2.8)
      ( 0.0, 3.2) node[above]{$C$}
      % Resistor
      ( 0.4, 3.0) -- ( 0.7, 3.0)
      ( 1.2, 3.0) -- ( 1.5, 3.0)
      ( 0.7, 3.1) -- ( 1.2, 3.1) -- ( 1.2, 2.9) -- ( 0.7, 2.9) -- cycle
      ( 0.95, 3.2) node[above]{$R$};
  \draw[black,thick,-*] ( 1.5, 3.0) -- ( 1.8, 3.0);
% Wire
\end{tikzpicture}\end{center}

        Functionality analysis:

        \begin{center}\begin{tikzpicture}
            \draw[black,thick]
                  (0.0,3.0) node[left]{$_{D_{IN}}$}
                --(0.0,3.0)
                --(1.0,3.0)--(1.0,3.6)
                --(2.0,3.6)--(2.0,3.0)
                --(12.0,3.0);
            \draw[black,thick]
                  (0.0,2.0) node[left]{$_{Q_{1}}$}
                --(0.0,2.0)
                --(1.0,2.0)--(1.0,2.6)
                --(7.0,2.6)--(7.0,2.0)
                --(12.0,2.0);
            \draw[black,thick]
                  (0.0,1.0) node[left]{$_{Q_{2}}$}
                --(0.0,1.0)
                --(2.0,1.0)--(2.0,1.6)
                --(8.0,1.6)--(8.0,1.0)
                --(12.0,1.0);
            \draw[black,thick]
                  (0.0,0.0) node[left]{$_{D_{OUT}}$}
                --(0.0,0.0)
                --(7.0,0.0)--(7.0,0.6)
                --(8.0,0.6)--(8.0,0.0)
                --(12.0,0.0);
            \draw[gray,dash pattern={on 0.015cm off 0.03cm},thick]
                (1.0,3.0) -- (1.0,-1.0)
                (2.0,3.0) -- (2.0,-1.0)
                (7.0,2.0) -- (7.0,-1.0)
                (8.0,1.0) -- (8.0,-1.0)
            ;
            \draw[gray,thick,<->]
                (1.0,-1.0) --(1.5,-1.0) node[below]{$t_W$} -- (2.0,-1.0);
            \draw[gray,thick,<->]
                (7.0,-1.0) --(7.5,-1.0) node[below]{$t_W$} -- (8.0,-1.0);
            \draw[hwSolution,thick,<->]
                (1.1, 2.3) --(4.0, 2.3) node[below]{$t_{CR}$} -- (6.9, 2.3);
            \draw[hwSolution,thick,<->]
                (2.1, 1.3) --(5.0, 1.3) node[below]{$t_{CR}$} -- (7.9, 1.3);
        \end{tikzpicture}\end{center}
    }
    \subsection{Homework}
    \subsubsection{8.3 \textnormal{Analyze Logical function of the given circuit}.}
        \begin{center}\begin{tikzpicture}
    \def \HFH{10}
    \coordinate (DATA) at (5, 0);
    \draw[black]
        \foreach \x in {0,...,7}{
            (DATA)+({\x * 0.2},\HFH)--+({\x * 0.2},-\HFH)
        }
        (DATA)+(0.0,\HFH) node[above,left]{$D_7$}
        (DATA)+(1.4,\HFH) node[above,right]{$D_0$};
    \coordinate (ADDR) at (0, 0);
    \draw[black]
        \foreach \x in {2,...,11}{
            (ADDR)+({\x * 0.1},\HFH)--+({\x * 0.1},-\HFH)
        }
        (ADDR)+(0.0,\HFH) node[above,left]{$A_{11}$}
        (ADDR)+(1.1,\HFH) node[above,right]{$A_0$};
    % 24 Decoder
    \coordinate (C0) at (-4, 0);
    \draw[hwSolution,thick]
        (C0)+(-1,1.6) -- +(1,1.6) -- +(1,-1.6) -- +(-1,-1.6) -- cycle;
    \draw[black]
        (C0)+(-1.0, 0.8) node[right]{$A_0$} -- +(-1.4, 0.8)
        (C0)+(-1.0,-0.8) node[right]{$A_1$} -- +(-1.4,-0.8)
        (C0)+(-1.4, 0.8) -- +(-1.4,{\HFH-1.5}) -- (0.1,{\HFH-1.5}) -- (0.1,\HFH)
        (C0)+(-1.4,-0.8) -- +(-1.5,-0.8) -- +(-1.5,{\HFH-1.4}) -- (0.0,{\HFH-1.4}) -- (0.0,\HFH);
    \draw[hwSolution,thick,o-*] (C0)+( 1.0, 1.2) node[left] {$Y_0$} -- +( 1.4, 1.2) --  (-2.4, 1.2) -- (-2.4, 6.7) -- (-0.9, 6.7);
    \draw[hwSolution,thick,o-*] (C0)+( 1.0, 0.4) node[left] {$Y_1$} -- +( 1.4, 0.4) --  (-2.0, 0.4) -- (-2.0, 1.9) -- (-0.9, 1.9);
    \draw[hwSolution,thick,o-*] (C0)+( 1.0,-0.4) node[left] {$Y_2$} -- +( 1.4,-0.4) --  (-2.0,-0.4) -- (-2.0,-2.9) -- (-0.9,-2.9);
    \draw[hwSolution,thick,o-*] (C0)+( 1.0,-1.2) node[left] {$Y_3$} -- +( 1.4,-1.2) --  (-2.4,-1.2) -- (-2.4,-7.7) -- (-0.9,-7.7);

    \foreach \Chip in {0,...,7}{
        \def \DWXSHIFT {2}
        \ifodd\Chip \def \DWXSHIFT{2.8} \fi
        \coordinate (C\Chip) at (3, {(3.5-\Chip)*2.4});
        \draw[hwSolution,thick]
            (C\Chip)+(-0.8,1) -- +(0.8,1) -- +(0.8,-1) -- +(-0.8,-1) -- cycle;
        \draw[black]
            (C\Chip)+(-0.9,0.8) node(A9C\Chip)[right]{$_{A_9}$}
            (C\Chip)+(-0.9,0.1) node(A0C\Chip)[right]{$_{A_0}$}
            \foreach \ADDRWIRE in {0,...,9}{
                (C\Chip)+(-0.8,{1 - 0.1*(1+\ADDRWIRE)}) -- +({\ADDRWIRE*0.1 - 2.8},{1 - 0.1*(1+\ADDRWIRE)})
            }
            \foreach \DATAWIRE in {0,...,3}{
                (C\Chip)+( 0.8,{0.3-0.2*\DATAWIRE}) node[left,xshift = 3]{$_{_{D_\DATAWIRE}}$} -- +({\DATAWIRE*-0.2 + 0.6 + \DWXSHIFT},{0.3-0.2*\DATAWIRE})
            };
        \draw[black,dotted,thick]
            (A9C\Chip) -- (A0C\Chip);
        \draw[hwSolution,thick,o-]
            (C\Chip)+(-0.8,-0.5) node[right,xshift = -2]{$_{\overline{CS}}$}
            --+(-4,-0.5)
            \ifodd\Chip --+(-4,1.9) \fi;
    }
\end{tikzpicture}\end{center}
    {\color{hwSolution}
    \begin{center}
    $K\equiv 1$,~~J=1 flip,~~J=0 reset.

    \begin{tabular}{|c|ccc|ccc|c|}
        \hline
        $CP $ & $Q_0$ & $Q_1$ & $Q_2$ & $Q_0^N$ & $Q_1^N$ & $Q_2^N$ & $CP^N$
        \\
        \hline
        $ 0 $ & $ 0 $ & $ 0 $ & $ 0 $ & $  1  $ & $  1  $ & $  0  $ & $ 6 $\\
        $ 1 $ & $ 0 $ & $ 0 $ & $ 1 $ & $  0  $ & $  0  $ & $  0  $ & $ 0 $\\
        $ 2 $ & $ 0 $ & $ 1 $ & $ 0 $ & $  1  $ & $  0  $ & $  0  $ & $ 4 $\\
        $ 3 $ & $ 0 $ & $ 1 $ & $ 1 $ & $  0  $ & $  1  $ & $  0  $ & $ 2 $\\
        $ 4 $ & $ 1 $ & $ 0 $ & $ 0 $ & $  0  $ & $  0  $ & $  0  $ & $ 0 $\\
        $ 5 $ & $ 1 $ & $ 0 $ & $ 1 $ & $  0  $ & $  0  $ & $  0  $ & $ 0 $\\
        $ 6 $ & $ 1 $ & $ 1 $ & $ 0 $ & $  0  $ & $  1  $ & $  1  $ & $ 3 $\\
        $ 7 $ & $ 1 $ & $ 1 $ & $ 1 $ & $  0  $ & $  1  $ & $  0  $ & $ 2 $\\ 
        \hline      
    \end{tabular}

    \begin{tikzpicture}
        \draw[hwSolution,thick]
        (0,3)node[above]{~}
        (0,3)node[below]{Carno Chart:}
        (-0.3,2) node(n5)[circle,draw]{5} 
        (0,1) node(n0)[circle,draw,]{0}
        (1.1, 0.6) node(n6)[circle,draw,]{6}
        (1.1,-0.6) node(n3)[circle,draw,]{3}
        (-0.8,0) node(n4)[circle,draw,]{4}
        (0,-1) node(n2)[circle,draw,]{2}
        (-0.3,-2) node(n7)[circle,draw,]{7}
        ;
        \draw [hwSolution,->,thick] (n5) -- (n0);
        \draw [hwSolution,->,thick] (n0) -- (n6);
        \draw [hwSolution,->,thick] (n6) -- (n3);
        \draw [hwSolution,->,thick] (n3) -- (n2);
        \draw [hwSolution,->,thick] (n2) -- (n4);
        \draw [hwSolution,->,thick] (n4) -- (n0);
        \draw [hwSolution,->,thick] (n7) -- (n2);
    \end{tikzpicture}

    \end{center}
    }
 
    \subsubsection{8.6 \textnormal{Design a circuit using Jump-Key flip-flop to serve given function}.}
    {\color{hwSolution}

        \begin{center}\begin{tikzpicture}
% JKFF CHIP 1 (-4.0)
    \draw[hwSolution,thick   ] (-4.8,2.4)--(-4.8,0.0) --(-3.2,0.0)--(-3.2,2.4)--cycle;
    % Left Side
    \draw[hwSolution,thick, -] (-4.8,2.0) node[right]{$1J$} -- (-5.5,2.0);
    \draw[hwSolution,thick,o-] (-4.8,1.2) node[right]{$~~~C1$} -- (-5.5,1.2);
    \draw[hwSolution,thick, -] (-4.8,0.4) node[right]{$1K$} -- (-5.5,0.4);
    %NEGEDGE
    \draw[hwSolution,thick, -] (-4.8,1.0) -- (-4.5,1.2) -- (-4.8,1.4) ;
    % Right Side
    \draw[hwSolution,thick,o-] (-3.2,0.6) node[left]{$\bar{Y}$} -- (-2.6,0.6);
    \draw[hwSolution,thick,- ] (-3.2,1.8) node[left]{$Y$} -- (-2.0,1.8) -- (-2.0,1.2) -- (-1.5,1.2);
% JKFF CHIP 2 (+0.0)
    \draw[hwSolution,thick   ] (-0.8,2.4)--(-0.8,0.0) --( 0.8,0.0)--( 0.8,2.4)--cycle;
    % Left Side
    \draw[hwSolution,thick, -] (-0.8,2.0) node[right]{$1J$} -- (-1.5,2.0);
    \draw[hwSolution,thick,o-] (-0.8,1.2) node[right]{$~~~C1$} -- (-1.5,1.2);
    \draw[hwSolution,thick, -] (-0.8,0.4) node[right]{$1K$} -- (-1.5,0.4);
    %NEGEDGE
    \draw[hwSolution,thick, -] (-0.8,1.0) -- (-0.5,1.2) -- (-0.8,1.4) ;
    % Right Side
    \draw[hwSolution,thick,o-] (0.8,0.6) node[left]{$\bar{Y}$} -- ( 1.4,0.6);
    \draw[hwSolution,thick,- ] (0.8,1.8) node[left]{$Y$} -- ( 2.0,1.8) -- ( 2.0,1.2) -- ( 2.5,1.2);
% JKFF CHIP 3 (+4.0)
    \draw[hwSolution,thick   ] ( 3.2,2.4)--( 3.2,0.0) --( 4.8,0.0)--( 4.8,2.4)--cycle;
    % Left Side
    \draw[hwSolution,thick, -] ( 3.2,2.0) node[right]{$1J$} -- ( 2.5,2.0);
    \draw[hwSolution,thick,o-] ( 3.2,1.2) node[right]{$~~~C1$} -- ( 2.5,1.2);
    \draw[hwSolution,thick, -] ( 3.2,0.4) node[right]{$1K$} -- ( 2.5,0.4);
    %NEGEDGE
    \draw[hwSolution,thick, -] ( 3.2,1.0) -- ( 3.5,1.2) -- ( 3.2,1.4) ;
    % Right Side
    \draw[hwSolution,thick,o-] ( 4.8,0.6) node[left]{$\bar{Y}$} -- ( 5.4,0.6);
    \draw[hwSolution,thick, -] ( 4.8,1.8) node[left]{$Y$} -- (5.5,1.8);
% NOTGATE CHIP * (+4.0)
    \draw[hwSolution,thick   ] ( 3.6,3.0)--( 4.2,3.3)--( 4.2,2.7)--cycle;
    % Wire CHIP <- Chip3 Y
    \draw[hwSolution,thick, -] ( 5.5,1.8) -- ( 5.5,3.0) -- ( 4.2,3.0);
    % Wire CHIP -> Chip* 1J
    \draw[hwSolution,thick,o-] ( 3.6,3.0) -- (-5.5,3.0) -- (-5.5,2.0);
    \draw[hwSolution,thick,*-] (-1.5,3.1) -- (-1.5,2.0);
    \draw[hwSolution,thick,*-] ( 2.5,3.1) -- ( 2.5,2.0);
% ANDGATE CHIP * (+4.0)
    \draw[hwSolution,thick   ] ( 3.6,-1.2)--( 4.2,-1.2)--( 4.2,-0.4)--( 3.6,-0.4)--cycle ( 3.9,-0.8)node{$\&$};
    % Wire - CHIP <- Chip* #Y
    \draw[hwSolution,thick, -] (-2.6, 0.6) -- (-2.6,-1.0) -- ( 3.6,-1.0);
    \draw[hwSolution,thick, -] ( 1.4, 0.6) -- ( 1.4,-0.8) -- ( 3.6,-0.8);
    \draw[hwSolution,thick, -] ( 5.4, 0.6) -- ( 5.4,-0.2) -- ( 3.2,-0.2) -- ( 3.2,-0.6) -- ( 3.6,-0.6);
    % Wire - CHIP -> Z
    \draw[hwSolution,thick, -] ( 4.2,-0.8) -- ( 5.2,-0.8) node[right]{Z};
% Wire - CP 
    % CP -> Chip1(#C1)
    \draw[hwSolution,thick, -] (-6.5,1.2) node[left]{$CP$} -- (-5.5,1.2);
% Wire - Vcc
    % Vcc -> Chip*(1K)
    \draw[hwSolution,thick, -] (-6.5,-0.4) node[left]{$V_{cc}$} -- ( 2.5,-0.4) -- ( 2.5, 0.4);
    \draw[hwSolution,thick,*-] (-5.5,-0.5) -- (-5.5, 0.4);
    \draw[hwSolution,thick,*-] (-1.5,-0.5) -- (-1.5, 0.4);
\end{tikzpicture}\end{center}

    }

    \subsubsection{8.7 \textnormal{Build a 60 counter with 74LS293}.}
    {\color{hwSolution}

        \begin{center}\begin{tikzpicture}
% 74LS293 CHIP 2 (+0.0)
    \draw[hwSolution,thick   ] (0-1.8,2.0)--(0-1.8,0.0) --(0+1.8,0.0)--(0+1.8,2.0)--cycle
    (0,1.0)node{\small{74293}};
    % Top Side
    \draw[hwSolution,thick, -] (0-0.8, 2.0) node[below]{$_{R_1}$} -- (0-0.8, 2.4);
    \draw[hwSolution,thick, -] (0+0.8, 2.0) node[below]{$_{R_2}$} -- (0+0.8, 2.4);
    % Left Side
    \draw[hwSolution,thick,o-] (0-1.8, 1.3) node[right]{$~~~_{CP_0}$} -- (0-2.4, 1.3);
    \draw[hwSolution,thick,o-] (0-1.8, 0.7) node[right]{$~~~_{CP_1}$} -- (0-2.4, 0.7);
    % Bottom Side
    \draw[hwSolution,thick, -] (0-0.9, 0.0) node[above]{$_{Q_0}$} -- (0-0.9,-1.2);
    \draw[hwSolution,thick, -] (0-0.3, 0.0) node[above]{$_{Q_1}$} -- (0-0.3,-1.2);
    \draw[hwSolution,thick, -] (0+0.3, 0.0) node[above]{$_{Q_2}$} -- (0+0.3,-1.2);
    \draw[hwSolution,thick, -] (0+0.9, 0.0) node[above]{$_{Q_3}$} -- (0+0.9,-1.2);
    %NEGEDGE
    \draw[hwSolution,thick, -] (0-1.8, 1.3+0.2) -- (0-1.8+0.3, 1.3) -- (0-1.8, 1.3-0.2);
    \draw[hwSolution,thick, -] (0-1.8, 0.7+0.2) -- (0-1.8+0.3, 0.7) -- (0-1.8, 0.7-0.2);
    %Self Connection
    \draw[hwSolution,thick,-*] (0-2.4, 0.7) -- (0-2.4,-0.4) -- (0-0.8,-0.4);
% 74LS293 CHIP 2 (+0.0)
    \draw[hwSolution,thick   ] (6-1.8,2.0)--(6-1.8,0.0) --(6+1.8,0.0)--(6+1.8,2.0)--cycle
    (6,1.0)node{\small{74293}};
    % Top Side
    \draw[hwSolution,thick, -] (6-0.8, 2.0) node[below]{$_{R_1}$} -- (6-0.8, 2.4);
    \draw[hwSolution,thick, -] (6+0.8, 2.0) node[below]{$_{R_2}$} -- (6+0.8, 2.4);
    % Left Side
    \draw[hwSolution,thick,o-] (6-1.8, 1.3) node[right]{$~~~_{CP_0}$} -- (6-2.4, 1.3);
    \draw[hwSolution,thick,o-] (6-1.8, 0.7) node[right]{$~~~_{CP_1}$} -- (6-2.4, 0.7);
    % Bottom Side
    \draw[hwSolution,thick, -] (6-0.9, 0.0) node[above]{$_{Q_0}$} -- (6-0.9,-1.2);
    \draw[hwSolution,thick, -] (6-0.3, 0.0) node[above]{$_{Q_1}$} -- (6-0.3,-1.2);
    \draw[hwSolution,thick, -] (6+0.3, 0.0) node[above]{$_{Q_2}$} -- (6+0.3,-1.2);
    \draw[hwSolution,thick, -] (6+0.9, 0.0) node[above]{$_{Q_3}$} -- (6+0.9,-1.2);
    %NEGEDGE
    \draw[hwSolution,thick, -] (6-1.8, 1.3+0.2) -- (6-1.8+0.3, 1.3) -- (6-1.8, 1.3-0.2);
    \draw[hwSolution,thick, -] (6-1.8, 0.7+0.2) -- (6-1.8+0.3, 0.7) -- (6-1.8, 0.7-0.2);
    %Self Connection
    \draw[hwSolution,thick,-*] (6-2.4, 0.7) -- (6-2.4,-0.4) -- (6-0.8,-0.4);
% ANDGATE CHIP *
    \draw[hwSolution,thick   ] (2.6-0.5,2.4)--(2.6-0.5,1.8) --(2.6+0.5,1.8)--(2.6+0.5,2.4)--cycle
    (2.6,2.1)node{$\&$};
% Wire
    % SET0 Wire Collection
    \draw[hwSolution,thick,*-] (6-0.9+0.1,-0.8) -- (2.6+0.1,-0.8) -- (2.6+0.1, 1.8); % 2_Q0
    \draw[hwSolution,thick,*-] (6-0.3+0.1,-0.6) -- (2.6+0.3,-0.6) -- (2.6+0.3, 1.8); % 2_Q1
    \draw[hwSolution,thick,*-] (0+0.3-0.1,-0.4) -- (2.6-0.1,-0.4) -- (2.6-0.1, 1.8); % 1_Q2
    \draw[hwSolution,thick,*-] (0+0.9-0.1,-0.2) -- (2.6-0.3,-0.2) -- (2.6-0.3, 1.8); % 1_Q3
    % Carry Bit Signal
    \draw[hwSolution,thick,*-] (2.6-0.3-0.1, 1.3) -- (6-2.4, 1.3);
    % Reset Signal
    \draw[hwSolution,thick, -]
    (0-0.8, 2.4) -- (0-0.8, 2.8) -- (2.6, 2.8)
    (0+0.8, 2.4) -- (0+0.8, 2.8)
    (6-0.8, 2.4) -- (6-0.8, 2.8)
    (6+0.8, 2.4) -- (6+0.8, 2.8) -- (2.6, 2.8)
    (2.6,2.4) -- (2.6, 2.8)
    ;
    % CP input
    \draw[hwSolution,thick, -] (0-2.4, 1.3) -- (0-2.8, 1.3) node[left]{$_{CP}$};
\end{tikzpicture}\end{center}

    }

    \subsubsection{8.12 \textnormal{Analyze Logical function of the given circuit}.}
    {\color{hwSolution}

        The given circuit forms a \textbf{\textit{196 counter}} with initial value \textbf{\textit{60}} and ending value \textbf{\textit{255}}.

    }

    \subsubsection{8.13 \textnormal{Give the value saved in each register according to given waveform}.}
    {\color{hwSolution}

    \begin{center}\begin{tabular}{|c|c|c|c|c|}
        \hline
        $ CP $ & $Reg_1$ & $Reg_2$ & $Reg_3$ & Action(s)
        \\
        \hline
        $ t0 $ & $1011$ & $1000$ & $0111$ & Chip Initialized \\
        $ t1 $ & $1011$ & $1000$ & \color{red}{$1000$} & MOV R3,R2 \\
        $ t2 $ & $1011$ & $1000$ & $1000$ & Enable R3 (No ST) \\
        $ t3 $ & $1011$ & $1000$ & $1000$ & No action \\
        $ t4 $ & $1011$ & \color{red}{$1011$} & \color{red}{$1011$} & MOV \{R2,R3\},R1 \\
        \hline
    \end{tabular}\end{center}

    }

    \subsubsection{8.17 \textnormal{ Given dseign of logical circuit: }}
    \noindent{1.List state sequence of the circuit, with initial state 0110}
    {\color{hwSolution}

    \begin{center}\begin{tabular}{|c|cccc|c|}
        \hline
        $ CP $ & $Q_0$ & $Q_1$ & $Q_2$ & $Q_3$ & Note
        \\
        \hline
        $ 0 $ & $0$ & $1$ & $1$ & $0$ & Initial \\
        $ 1 $ & $0$ & $0$ & $1$ & $1$ & $\rightarrow$ \\
        $ 2 $ & $1$ & $0$ & $0$ & $1$ & $\rightarrow$\\
        $ 3 $ & $1$ & $1$ & $0$ & $0$ & $\rightarrow$\\
        $ 4 $ & $0$ & $1$ & $1$ & $0$ & Repeat $CP_0$\\
        \hline
    \end{tabular}\end{center}

    }

    \noindent{2.List the output secquence of L}
    {\color{hwSolution}

    \begin{center}\begin{tabular}{|c|cccc|c|c|}
        \hline
        $ CP $ & $Q_0$ & $Q_1$ & $Q_2$ & $Q_3$ & $D_{Selected}$ & Y
        \\
        \hline
        $ 0 $ & $0$ & $1$ & $1$ & $0$ & $D_{3}$ & 0 \\
        $ 1 $ & $0$ & $0$ & $1$ & $1$ & $D_{1}$ & 1 \\
        $ 2 $ & $1$ & $0$ & $0$ & $1$ & $D_{4}$ & 0 \\
        $ 3 $ & $1$ & $1$ & $0$ & $0$ & $D_{6}$ & 0 \\
        \hline
    \end{tabular}\end{center}

    }