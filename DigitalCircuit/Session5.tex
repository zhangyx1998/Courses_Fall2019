\section{Session 5 - Sequential circuit and unit}
\vspace{-15pt}\noindent\rule{\textwidth}{0.1pt}\vspace{-10pt}
    \subsection*{Session 5 Notes}
    {\color{hwSolution}
    \subsection*{Another method to generate delayed paluse - without losing its length}
        \begin{center}\begin{tikzpicture}
% CHIP 2 (Y+0.0)
  \draw[black,thick]
    (-1.2,2.4)--(-1.2,0.0)
  --( 1.2,0.0)--( 1.2,2.4)
  --cycle;
  % Left Side
  \draw[black,thick, -] (-1.2,1.6) node[right]{$A$}
  -- (-2.0,1.6) node[left]{$V_{DD}$};
  \draw[black,thick,o-] (-1.2, 0.8) node[right]{$B$}
  -- (-2.0,0.8);
  % Right Side
  \draw[black,thick, -] ( 1.2, 1.6) node[left]{$Q$}
  -- ( 2.0, 1.6);
  \draw[black,thick,o-] ( 1.2, 0.8) node[left]{$\bar{Q}$}
  -- ( 1.4, 0.8);
  % Top Side
  \draw[black,thick, -] (-0.4, 2.4) node[below]{$C_X$}
  -- (-0.4, 3.0);
  \draw[black,thick, -] ( 0.4, 2.4) node[below]{$R_X$}
  -- ( 0.4, 3.0);
  % Bottom Side
  \draw[black,thick,o-*] ( 0.0, 0.0) node[above]{$C_{L}$}
  -- ( 0.0,-0.6) -- ( 1.8,-0.6);
  % C/R
  \draw[black,thick, -]
      % Capacitor
      (-0.4, 3.0) -- (-0.1, 3.0)
      ( 0.4, 3.0) -- ( 0.1, 3.0)
      ( 0.1, 3.2) -- ( 0.1, 2.8)
      (-0.1, 3.2) -- (-0.1, 2.8)
      ( 0.0, 3.2) node[above]{$C$}
      % Resistor
      ( 0.4, 3.0) -- ( 0.7, 3.0)
      ( 1.2, 3.0) -- ( 1.5, 3.0)
      ( 0.7, 3.1) -- ( 1.2, 3.1) -- ( 1.2, 2.9) -- ( 0.7, 2.9) -- cycle
      ( 0.95, 3.2) node[above]{$R$};
  \draw[black,thick,-*] ( 1.5, 3.0) -- ( 1.8, 3.0);
% CHIP 1 (Y+5.0)
    \draw[black,thick]
    (-1.2,7.4)--(-1.2,5.0)
  --( 1.2,5.0)--( 1.2,7.4)
  --cycle;
  % Left Side
  \draw[black,thick, -] (-1.2,6.6) node[right]{$A$}
  -- (-2.0,6.6);
  \draw[black,thick,o-] (-1.2, 5.8) node[right]{$B$}
  -- (-2.0,5.8) node[left]{\textit{GND}};
  % Right Side
  \draw[black,thick, -] ( 1.2, 6.6) node[left]{$Q$}
  -- ( 2.0, 6.6);
  \draw[black,thick,o-] ( 1.2, 5.8) node[left]{$\bar{Q}$}
  -- ( 1.4, 5.8);
  % Top Side
  \draw[black,thick, -] (-0.4, 7.4) node[below]{$C_X$}
  -- (-0.4, 8.0);
  \draw[black,thick, -] ( 0.4, 7.4) node[below]{$R_X$}
  -- ( 0.4, 8.0);
  % Bottom Side
  \draw[black,thick,o-*] ( 0.0, 5.0) node[above]{$C_{L}$}
  -- ( 0.0, 4.4) -- ( 1.8, 4.4);
  % C/R
  \draw[black,thick, -]
    % Capacitor
    (-0.4, 8.0) -- (-0.1, 8.0)
    ( 0.4, 8.0) -- ( 0.1, 8.0)
    ( 0.1, 8.2) -- ( 0.1, 7.8)
    (-0.1, 8.2) -- (-0.1, 7.8)
    ( 0.0, 8.2) node[above]{$C$}
    % Resistor
    ( 0.4, 8.0) -- ( 0.7, 8.0)
    ( 1.2, 8.0) -- ( 1.5, 8.0)
    ( 0.7, 8.1) -- ( 1.2, 8.1) -- ( 1.2, 7.9) -- ( 0.7, 7.9) -- cycle
    ( 0.95,8.2) node[above]{$R$};
  \draw[black,thick,-*] ( 1.5, 8.0) -- ( 1.8, 8.0);
% AND Gate (4.0,3.7)
  \draw[black,thick]
  (4.0,3.7) node{$\&$}
  (3.5,2.9)--(3.5,4.5)
  --(4.5,4.5)--(4.5,2.9)
  --cycle;
% Wire
  \draw[black,thick, -] ( 1.7, 9.0) node[above]{$V_{DD}$}
  -- ( 1.7,-0.6);
  \draw[black,thick, -] (-6.0, 3.7) node[left]{$D_{IN}$} -- (-4.0, 3.7)
  -- (-4.0,6.6) -- (-2.0,6.6)
  (-4.0, 3.7) -- (-4.0,0.8) -- (-2.0,0.8);
  \draw[black,thick, -]
  ( 2.0, 1.6) -- ( 3.0, 1.6) -- ( 3.0, 3.4) -- ( 3.5, 3.4)
  ( 2.0, 6.6) -- ( 3.0, 6.6) -- ( 3.0, 4.0) -- ( 3.5, 4.0);
  \draw[black,thick, -] (4.5,3.7) -- (6.0,3.7) node[right]{$D_{OUT}$};
\end{tikzpicture}\end{center}

    \textit{Functionality analysis:}

        \begin{center}\begin{tikzpicture}
            \draw[black,thick]
                  (0.0,3.0) node[left]{$_{D_{IN}}$}
                --(0.0,3.0)
                --(1.0,3.0)--(1.0,3.6)
                --(2.0,3.6)--(2.0,3.0)
                --(12.0,3.0);
            \draw[black,thick]
                  (0.0,2.0) node[left]{$_{Q_{1}}$}
                --(0.0,2.0)
                --(1.0,2.0)--(1.0,2.6)
                --(7.0,2.6)--(7.0,2.0)
                --(12.0,2.0);
            \draw[black,thick]
                  (0.0,1.0) node[left]{$_{Q_{2}}$}
                --(0.0,1.0)
                --(2.0,1.0)--(2.0,1.6)
                --(8.0,1.6)--(8.0,1.0)
                --(12.0,1.0);
            \draw[black,thick]
                  (0.0,0.0) node[left]{$_{D_{OUT}}$}
                --(0.0,0.0)
                --(7.0,0.0)--(7.0,0.6)
                --(8.0,0.6)--(8.0,0.0)
                --(12.0,0.0);
            \draw[gray,dash pattern={on 0.015cm off 0.03cm},thick]
                (1.0,3.0) -- (1.0,-1.0)
                (2.0,3.0) -- (2.0,-1.0)
                (7.0,2.0) -- (7.0,-1.0)
                (8.0,1.0) -- (8.0,-1.0)
            ;
            \draw[gray,thick,<->]
                (1.0,-1.0) --(1.5,-1.0) node[below]{$t_W$} -- (2.0,-1.0);
            \draw[gray,thick,<->]
                (7.0,-1.0) --(7.5,-1.0) node[below]{$t_W$} -- (8.0,-1.0);
            \draw[hwSolution,thick,<->]
                (1.1, 2.3) --(4.0, 2.3) node[below]{$t_{CR}$} -- (6.9, 2.3);
            \draw[hwSolution,thick,<->]
                (2.1, 1.3) --(5.0, 1.3) node[below]{$t_{CR}$} -- (7.9, 1.3);
        \end{tikzpicture}\end{center}
    }
    \subsection*{Session 5 Homework}
    \subsubsection{8.3 \textnormal{Analyze Logical function of the given circuit}.}
        \begin{center}\begin{tikzpicture}
    \draw[black,thick,->] (-0.5,0) -- (10,0) node[below]{$t$};
    \draw[black,thick,->] (0,-0.5) -- ( 0,3.5) node[left]{$u_I$};
    \draw[black,dash pattern={on 0.2cm off 0.2cm}]
    (0,1) node[left]{$U_{T-}$} -- (9.5,1)
    (0,2) node[left]{$U_{T+}$} -- (9.5,2);
    \draw[gray,thick] plot[smooth] coordinates{
    (0,0)
    (0.5,2.0) %N1
    (0.7,2.2)
    (1.1,1.8)
    (1.4,2.1)
    (1.6,1.8)
    (1.9,2.1)
    (2.2,1.0) %N2
    (2.3,0.0)
    (2.8,1.0)
    (3.0,1.3)
    (3.3,1.4)
    (3.5,0.8)
    (3.7,1.2)
    (4.1,0.7)
    (4.6,2.0) %N3
    (5.1,2.3)
    (5.3,1.8)
    (5.5,1.8)
    (5.9,2.6)
    (6.2,1.0) %N4
    (6.6,0.0)
    (7.0,1.3)
    (7.6,0.8)
    (7.9,1.3)
    (8.4,0.8)
    (9.0,1.2)};
    \draw[hwSolution,thick](0,0.5) node[left,hwSolution]{$u_0$}
    -- (0.5,0.5) -- (0.5,2.5)
    -- (2.2,2.5) -- (2.2,0.5)
    -- (4.6,0.5) -- (4.6,2.5)
    -- (6.2,2.5) -- (6.2,0.5)
    -- (9.0,0.5);
\end{tikzpicture}\end{center}
    {\color{hwSolution}
    \begin{center}
    $K\equiv 1$,~~J=1 flip,~~J=0 reset.

    \begin{tabular}{|c|ccc|ccc|c|}
        \hline
        $CP $ & $Q_0$ & $Q_1$ & $Q_2$ & $Q_0^N$ & $Q_1^N$ & $Q_2^N$ & $CP^N$
        \\
        \hline
        $ 0 $ & $ 0 $ & $ 0 $ & $ 0 $ & $  1  $ & $  1  $ & $  0  $ & $ 6 $\\
        $ 1 $ & $ 0 $ & $ 0 $ & $ 1 $ & $  0  $ & $  0  $ & $  0  $ & $ 0 $\\
        $ 2 $ & $ 0 $ & $ 1 $ & $ 0 $ & $  1  $ & $  0  $ & $  0  $ & $ 4 $\\
        $ 3 $ & $ 0 $ & $ 1 $ & $ 1 $ & $  0  $ & $  1  $ & $  0  $ & $ 2 $\\
        $ 4 $ & $ 1 $ & $ 0 $ & $ 0 $ & $  0  $ & $  0  $ & $  0  $ & $ 0 $\\
        $ 5 $ & $ 1 $ & $ 0 $ & $ 1 $ & $  0  $ & $  0  $ & $  0  $ & $ 0 $\\
        $ 6 $ & $ 1 $ & $ 1 $ & $ 0 $ & $  0  $ & $  1  $ & $  1  $ & $ 3 $\\
        $ 7 $ & $ 1 $ & $ 1 $ & $ 1 $ & $  0  $ & $  1  $ & $  0  $ & $ 2 $\\ 
        \hline      
    \end{tabular}

    \begin{tikzpicture}
        \draw[hwSolution,thick]
        (0,3)node[above]{~}
        (0,3)node[below]{Carno Chart:}
        (-0.3,2) node(n5)[circle,draw]{5} 
        (0,1) node(n0)[circle,draw,]{0}
        (1.1, 0.6) node(n6)[circle,draw,]{6}
        (1.1,-0.6) node(n3)[circle,draw,]{3}
        (-0.8,0) node(n4)[circle,draw,]{4}
        (0,-1) node(n2)[circle,draw,]{2}
        (-0.3,-2) node(n7)[circle,draw,]{7}
        ;
        \draw [hwSolution,->,thick] (n5) -- (n0);
        \draw [hwSolution,->,thick] (n0) -- (n6);
        \draw [hwSolution,->,thick] (n6) -- (n3);
        \draw [hwSolution,->,thick] (n3) -- (n2);
        \draw [hwSolution,->,thick] (n2) -- (n4);
        \draw [hwSolution,->,thick] (n4) -- (n0);
        \draw [hwSolution,->,thick] (n7) -- (n2);
    \end{tikzpicture}

    \end{center}
    }
 
    \subsubsection{8.6 \textnormal{Design a circuit using Jump-Key flip-flop to serve given function}.}
    {\color{hwSolution}

        \begin{center}\begin{tikzpicture}
% JKFF CHIP 1 (-4.0)
    \draw[hwSolution,thick   ] (-4.8,2.4)--(-4.8,0.0) --(-3.2,0.0)--(-3.2,2.4)--cycle;
    % Left Side
    \draw[hwSolution,thick, -] (-4.8,2.0) node[right]{$1J$} -- (-5.5,2.0);
    \draw[hwSolution,thick,o-] (-4.8,1.2) node[right]{$~~~C1$} -- (-5.5,1.2);
    \draw[hwSolution,thick, -] (-4.8,0.4) node[right]{$1K$} -- (-5.5,0.4);
    %NEGEDGE
    \draw[hwSolution,thick, -] (-4.8,1.0) -- (-4.5,1.2) -- (-4.8,1.4) ;
    % Right Side
    \draw[hwSolution,thick,o-] (-3.2,0.6) node[left]{$\bar{Y}$} -- (-2.6,0.6);
    \draw[hwSolution,thick,- ] (-3.2,1.8) node[left]{$Y$} -- (-2.0,1.8) -- (-2.0,1.2) -- (-1.5,1.2);
% JKFF CHIP 2 (+0.0)
    \draw[hwSolution,thick   ] (-0.8,2.4)--(-0.8,0.0) --( 0.8,0.0)--( 0.8,2.4)--cycle;
    % Left Side
    \draw[hwSolution,thick, -] (-0.8,2.0) node[right]{$1J$} -- (-1.5,2.0);
    \draw[hwSolution,thick,o-] (-0.8,1.2) node[right]{$~~~C1$} -- (-1.5,1.2);
    \draw[hwSolution,thick, -] (-0.8,0.4) node[right]{$1K$} -- (-1.5,0.4);
    %NEGEDGE
    \draw[hwSolution,thick, -] (-0.8,1.0) -- (-0.5,1.2) -- (-0.8,1.4) ;
    % Right Side
    \draw[hwSolution,thick,o-] (0.8,0.6) node[left]{$\bar{Y}$} -- ( 1.4,0.6);
    \draw[hwSolution,thick,- ] (0.8,1.8) node[left]{$Y$} -- ( 2.0,1.8) -- ( 2.0,1.2) -- ( 2.5,1.2);
% JKFF CHIP 3 (+4.0)
    \draw[hwSolution,thick   ] ( 3.2,2.4)--( 3.2,0.0) --( 4.8,0.0)--( 4.8,2.4)--cycle;
    % Left Side
    \draw[hwSolution,thick, -] ( 3.2,2.0) node[right]{$1J$} -- ( 2.5,2.0);
    \draw[hwSolution,thick,o-] ( 3.2,1.2) node[right]{$~~~C1$} -- ( 2.5,1.2);
    \draw[hwSolution,thick, -] ( 3.2,0.4) node[right]{$1K$} -- ( 2.5,0.4);
    %NEGEDGE
    \draw[hwSolution,thick, -] ( 3.2,1.0) -- ( 3.5,1.2) -- ( 3.2,1.4) ;
    % Right Side
    \draw[hwSolution,thick,o-] ( 4.8,0.6) node[left]{$\bar{Y}$} -- ( 5.4,0.6);
    \draw[hwSolution,thick, -] ( 4.8,1.8) node[left]{$Y$} -- (5.5,1.8);
% NOTGATE CHIP * (+4.0)
    \draw[hwSolution,thick   ] ( 3.6,3.0)--( 4.2,3.3)--( 4.2,2.7)--cycle;
    % Wire CHIP <- Chip3 Y
    \draw[hwSolution,thick, -] ( 5.5,1.8) -- ( 5.5,3.0) -- ( 4.2,3.0);
    % Wire CHIP -> Chip* 1J
    \draw[hwSolution,thick,o-] ( 3.6,3.0) -- (-5.5,3.0) -- (-5.5,2.0);
    \draw[hwSolution,thick,*-] (-1.5,3.1) -- (-1.5,2.0);
    \draw[hwSolution,thick,*-] ( 2.5,3.1) -- ( 2.5,2.0);
% ANDGATE CHIP * (+4.0)
    \draw[hwSolution,thick   ] ( 3.6,-1.2)--( 4.2,-1.2)--( 4.2,-0.4)--( 3.6,-0.4)--cycle ( 3.9,-0.8)node{$\&$};
    % Wire - CHIP <- Chip* #Y
    \draw[hwSolution,thick, -] (-2.6, 0.6) -- (-2.6,-1.0) -- ( 3.6,-1.0);
    \draw[hwSolution,thick, -] ( 1.4, 0.6) -- ( 1.4,-0.8) -- ( 3.6,-0.8);
    \draw[hwSolution,thick, -] ( 5.4, 0.6) -- ( 5.4,-0.2) -- ( 3.2,-0.2) -- ( 3.2,-0.6) -- ( 3.6,-0.6);
    % Wire - CHIP -> Z
    \draw[hwSolution,thick, -] ( 4.2,-0.8) -- ( 5.2,-0.8) node[right]{Z};
% Wire - CP 
    % CP -> Chip1(#C1)
    \draw[hwSolution,thick, -] (-6.5,1.2) node[left]{$CP$} -- (-5.5,1.2);
% Wire - Vcc
    % Vcc -> Chip*(1K)
    \draw[hwSolution,thick, -] (-6.5,-0.4) node[left]{$V_{cc}$} -- ( 2.5,-0.4) -- ( 2.5, 0.4);
    \draw[hwSolution,thick,*-] (-5.5,-0.5) -- (-5.5, 0.4);
    \draw[hwSolution,thick,*-] (-1.5,-0.5) -- (-1.5, 0.4);
\end{tikzpicture}\end{center}

    }

    \subsubsection{8.7 \textnormal{Build a 60 counter with 74LS293}.}
    {\color{hwSolution}

        \begin{center}\begin{tikzpicture}
% 74LS293 CHIP 2 (+0.0)
    \draw[hwSolution,thick   ] (0-1.8,2.0)--(0-1.8,0.0) --(0+1.8,0.0)--(0+1.8,2.0)--cycle (0,1.0)node{\small{74293}};
    % Top Side
    \draw[hwSolution,thick, -] (0-0.8, 2.0) node[below]{$_{R_1}$} -- (0-0.8, 2.4);
    \draw[hwSolution,thick, -] (0+0.8, 2.0) node[below]{$_{R_2}$} -- (0+0.8, 2.4);
    % Left Side
    \draw[hwSolution,thick,o-] (0-1.8, 1.3) node[right]{$~~~_{CP_0}$} -- (0-2.4, 1.3);
    \draw[hwSolution,thick,o-] (0-1.8, 0.7) node[right]{$~~~_{CP_1}$} -- (0-2.4, 0.7);
    % Bottom Side
    \draw[hwSolution,thick, -] (0-1.2, 0.0) node[above]{$_{Q_0}$} -- (0-1.2,-0.8) node[below]{$_{Q_0}$};
    \draw[hwSolution,thick, -] (0-0.4, 0.0) node[above]{$_{Q_1}$} -- (0-0.4,-0.8) node[below]{$_{Q_1}$};
    \draw[hwSolution,thick, -] (0+0.4, 0.0) node[above]{$_{Q_2}$} -- (0+0.4,-0.8) node[below]{$_{Q_2}$};
    \draw[hwSolution,thick, -] (0+1.2, 0.0) node[above]{$_{Q_3}$} -- (0+1.2,-0.8) node[below]{$_{Q_3}$};
    %NEGEDGE
    \draw[hwSolution,thick, -] (0-1.8, 1.3+0.2) -- (0-1.8+0.3, 1.3) -- (0-1.8, 1.3-0.2);
    \draw[hwSolution,thick, -] (0-1.8, 0.7+0.2) -- (0-1.8+0.3, 0.7) -- (0-1.8, 0.7-0.2);
    %Self Connection
    \draw[hwSolution,thick,-*] (0-2.4, 0.7) -- (0-2.4,-0.4) -- (0-1.1,-0.4);
% 74LS293 CHIP 2 (+0.0)
    \draw[hwSolution,thick   ] (6-1.8,2.0)--(6-1.8,0.0) --(6+1.8,0.0)--(6+1.8,2.0)--cycle (6,1.0)node{\small{74293}};
    % Top Side
    \draw[hwSolution,thick, -] (6-0.8, 2.0) node[below]{$_{R_1}$} -- (6-0.8, 2.4);
    \draw[hwSolution,thick, -] (6+0.8, 2.0) node[below]{$_{R_2}$} -- (6+0.8, 2.4);
    % Left Side
    \draw[hwSolution,thick,o-] (6-1.8, 1.3) node[right]{$~~~_{CP_0}$} -- (6-2.4, 1.3);
    \draw[hwSolution,thick,o-] (6-1.8, 0.7) node[right]{$~~~_{CP_1}$} -- (6-2.4, 0.7);
    % Bottom Side
    \draw[hwSolution,thick, -] (6-1.2, 0.0) node[above]{$_{Q_0}$} -- (6-1.2,-0.8) node[below]{$_{Q_0}$};
    \draw[hwSolution,thick, -] (6-0.4, 0.0) node[above]{$_{Q_1}$} -- (6-0.4,-0.8) node[below]{$_{Q_1}$};
    \draw[hwSolution,thick, -] (6+0.4, 0.0) node[above]{$_{Q_2}$} -- (6+0.4,-0.8) node[below]{$_{Q_2}$};
    \draw[hwSolution,thick, -] (6+1.2, 0.0) node[above]{$_{Q_3}$} -- (6+1.2,-0.8) node[below]{$_{Q_3}$};
    %NEGEDGE
    \draw[hwSolution,thick, -] (6-1.8, 1.3+0.2) -- (6-1.8+0.3, 1.3) -- (6-1.8, 1.3-0.2);
    \draw[hwSolution,thick, -] (6-1.8, 0.7+0.2) -- (6-1.8+0.3, 0.7) -- (6-1.8, 0.7-0.2);
    %Self Connection
    \draw[hwSolution,thick,-*] (6-2.4, 0.7) -- (6-2.4,-0.4) -- (6-1.1,-0.4);
% Wire
\end{tikzpicture}\end{center}

    }

    \subsubsection{8.12 \textnormal{Analyze Logical function of the given circuit}.}
    {\color{hwSolution}

        The given circuit forms a \textbf{\textit{196 counter}} with initial value \textbf{\textit{60}} and ending value \textbf{\textit{255}}.

    }

    \subsubsection{8.13 \textnormal{Give the value saved in each register according to given waveform}.}
    {\color{hwSolution}

    \begin{center}\begin{tabular}{|c|c|c|c|c|}
        \hline
        $ CP $ & $Reg_1$ & $Reg_2$ & $Reg_3$ & Action(s)
        \\
        \hline
        $ t0 $ & $1011$ & $1000$ & $0111$ & Chip Initialized \\
        $ t1 $ & $1011$ & $1000$ & \color{red}{$1000$} & MOV R3,R2 \\
        $ t2 $ & $1011$ & $1000$ & $1000$ & Enable R3 (No ST) \\
        $ t3 $ & $1011$ & $1000$ & $1000$ & No action \\
        $ t4 $ & $1011$ & \color{red}{$1011$} & \color{red}{$1011$} & MOV \{R2,R3\},R1 \\
        \hline
    \end{tabular}\end{center}

    }

    \subsubsection{8.17 \textnormal{ Given dseign of logical circuit: }}
    \noindent{1.List state sequence of the circuit, with initial state 0110}
    {\color{hwSolution}

    \begin{center}\begin{tabular}{|c|cccc|c|}
        \hline
        $ CP $ & $Q_0$ & $Q_1$ & $Q_2$ & $Q_3$ & Note
        \\
        \hline
        $ 0 $ & $0$ & $1$ & $1$ & $0$ & Initial \\
        $ 1 $ & $0$ & $0$ & $1$ & $1$ & $\rightarrow$ \\
        $ 2 $ & $1$ & $0$ & $0$ & $1$ & $\rightarrow$\\
        $ 3 $ & $1$ & $1$ & $0$ & $0$ & $\rightarrow$\\
        $ 4 $ & $0$ & $1$ & $1$ & $0$ & Repeat $CP_0$\\
        \hline
    \end{tabular}\end{center}

    }

    \noindent{2.List the output secquence of L}
    {\color{hwSolution}

    \begin{center}\begin{tabular}{|c|cccc|c|c|}
        \hline
        $ CP $ & $Q_0$ & $Q_1$ & $Q_2$ & $Q_3$ & $D_{Selected}$ & Y
        \\
        \hline
        $ 0 $ & $0$ & $1$ & $1$ & $0$ & $D_{3}$ & 0 \\
        $ 1 $ & $0$ & $0$ & $1$ & $1$ & $D_{1}$ & 1 \\
        $ 2 $ & $1$ & $0$ & $0$ & $1$ & $D_{4}$ & 0 \\
        $ 3 $ & $1$ & $1$ & $0$ & $0$ & $D_{6}$ & 0 \\
        \hline
    \end{tabular}\end{center}

    }