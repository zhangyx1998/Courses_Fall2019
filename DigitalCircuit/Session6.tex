\section{Session 6 - Pulse generating and shaping}
\vspace{-15pt}\noindent\rule{\textwidth}{0.1pt}\vspace{-10pt}
    \subsection*{Session 6 Homework}
    \subsubsection{9.1 \textnormal{ Give the waveform of $u_0$ }.}
    {\color{hwSolution}

        \begin{center}\begin{tikzpicture}
    \def \HFH{10}
    \coordinate (DATA) at (5, 0);
    \draw[black]
        \foreach \x in {0,...,7}{
            (DATA)+({\x * 0.2},\HFH)--+({\x * 0.2},-\HFH)
        }
        (DATA)+(0.0,\HFH) node[above,left]{$D_7$}
        (DATA)+(1.4,\HFH) node[above,right]{$D_0$};
    \coordinate (ADDR) at (0, 0);
    \draw[black]
        \foreach \x in {2,...,11}{
            (ADDR)+({\x * 0.1},\HFH)--+({\x * 0.1},-\HFH)
        }
        (ADDR)+(0.0,\HFH) node[above,left]{$A_{11}$}
        (ADDR)+(1.1,\HFH) node[above,right]{$A_0$};
    % 24 Decoder
    \coordinate (C0) at (-4, 0);
    \draw[hwSolution,thick]
        (C0)+(-1,1.6) -- +(1,1.6) -- +(1,-1.6) -- +(-1,-1.6) -- cycle;
    \draw[black]
        (C0)+(-1.0, 0.8) node[right]{$A_0$} -- +(-1.4, 0.8)
        (C0)+(-1.0,-0.8) node[right]{$A_1$} -- +(-1.4,-0.8)
        (C0)+(-1.4, 0.8) -- +(-1.4,{\HFH-1.5}) -- (0.1,{\HFH-1.5}) -- (0.1,\HFH)
        (C0)+(-1.4,-0.8) -- +(-1.5,-0.8) -- +(-1.5,{\HFH-1.4}) -- (0.0,{\HFH-1.4}) -- (0.0,\HFH);
    \draw[hwSolution,thick,o-*] (C0)+( 1.0, 1.2) node[left] {$Y_0$} -- +( 1.4, 1.2) --  (-2.4, 1.2) -- (-2.4, 6.7) -- (-0.9, 6.7);
    \draw[hwSolution,thick,o-*] (C0)+( 1.0, 0.4) node[left] {$Y_1$} -- +( 1.4, 0.4) --  (-2.0, 0.4) -- (-2.0, 1.9) -- (-0.9, 1.9);
    \draw[hwSolution,thick,o-*] (C0)+( 1.0,-0.4) node[left] {$Y_2$} -- +( 1.4,-0.4) --  (-2.0,-0.4) -- (-2.0,-2.9) -- (-0.9,-2.9);
    \draw[hwSolution,thick,o-*] (C0)+( 1.0,-1.2) node[left] {$Y_3$} -- +( 1.4,-1.2) --  (-2.4,-1.2) -- (-2.4,-7.7) -- (-0.9,-7.7);

    \foreach \Chip in {0,...,7}{
        \def \DWXSHIFT {2}
        \ifodd\Chip \def \DWXSHIFT{2.8} \fi
        \coordinate (C\Chip) at (3, {(3.5-\Chip)*2.4});
        \draw[hwSolution,thick]
            (C\Chip)+(-0.8,1) -- +(0.8,1) -- +(0.8,-1) -- +(-0.8,-1) -- cycle;
        \draw[black]
            (C\Chip)+(-0.9,0.8) node(A9C\Chip)[right]{$_{A_9}$}
            (C\Chip)+(-0.9,0.1) node(A0C\Chip)[right]{$_{A_0}$}
            \foreach \ADDRWIRE in {0,...,9}{
                (C\Chip)+(-0.8,{1 - 0.1*(1+\ADDRWIRE)}) -- +({\ADDRWIRE*0.1 - 2.8},{1 - 0.1*(1+\ADDRWIRE)})
            }
            \foreach \DATAWIRE in {0,...,3}{
                (C\Chip)+( 0.8,{0.3-0.2*\DATAWIRE}) node[left,xshift = 3]{$_{_{D_\DATAWIRE}}$} -- +({\DATAWIRE*-0.2 + 0.6 + \DWXSHIFT},{0.3-0.2*\DATAWIRE})
            };
        \draw[black,dotted,thick]
            (A9C\Chip) -- (A0C\Chip);
        \draw[hwSolution,thick,o-]
            (C\Chip)+(-0.8,-0.5) node[right,xshift = -2]{$_{\overline{CS}}$}
            --+(-4,-0.5)
            \ifodd\Chip --+(-4,1.9) \fi;
    }
\end{tikzpicture}\end{center}
        
    }

    \subsubsection{9.5 \textnormal{Given a 74121 connected as shown}.}
    \noindent{1.Calculate the range of delay time}
    {\color{hwSolution}
        \[
            CR \leq t_d \leq C(R+R_W)~~\\
        \]
        \[
            3.57ms \leq t_d \leq 18.97ms
        \]
    }

    \noindent{2.What's the functionality of the resister next to $R_w$?}
    {\color{hwSolution}
    
        It prevents short circuit when $R_w$ is set to 0.
    
    }

    \subsubsection{9.8}
    \noindent{1. Analyze the status of circuit when S is open.}
    {\color{hwSolution}

        When S remains open, \[\overline{TR}\equiv V_{cc} > \frac{1}{3}V_{cc}\]

        TH will be flipped to Low if it was High, and remains Low as a stable status.

        Hence, $u_0$ holds on $0$. The circuit is stable.

    }

    \noindent{2. Let $C=10\,\mu F$, give the value of $R$ so as the circuit outputs a pulse of $t_w = 10s$ when S is pressed.}
    {\color{hwSolution}

        Since the given design is a standard monostable trigger, we can use $t_w = RC\,ln3 = 10s$.

        Hence, $R = 910\,k\Omega$.

    }

    \noindent{3. What's the value of $R$ if $C=0.1\,\mu F,~~t_w=5\,ms$? What value of $t_w$ do we expect if we replace $C$ by $1\,\mu F$ with the same $R$?}
    {\color{hwSolution}
        \begin{align*}
            t_w = RC\,ln3 &= 5\,ms\\
            R = \dfrac{5\,ms}{0.1\,\mu F\,\cdot\,ln3} &= 45.5\,k\Omega \\
            Replace \Rightarrow& C=1\,\mu F\\
            t_w &= 50\,ms
        \end{align*}
    }

    \subsubsection{9.13 \textnormal{}.}
    \noindent{1. What kind of function dose each of the 555 chip serve?}
    {\color{hwSolution}

        Each of them is a multivibrator.
    }

    \noindent{2. Analyze the status of circuit when S is set to 1.}
    {\color{hwSolution}

        Charging time:
        \begin{align*}
            (Chip1)~~T_{1} &= (R_1+R_2)C\,ln\,2 = 2.84\,ms\\
            (Chip2)~~T_{1} &= (R_1+R_2)C\,ln\,2 = 0.284\,ms
        \end{align*}

        Discharge time:
        \begin{align*}
            (Chip1)~~T_{2} &= R_2\,C\,ln\,2 = 1.53\,ms\\
            (Chip2)~~T_{2} &= R_2\,C\,ln\,2 = 0.153\,ms
        \end{align*}

    Duty cycle:
    \begin{align*}
        (Chip1)~~F &= 228.3\,Hz\\
        (Chip2)~~F &= 2.283\,kHz\\
        Ratio&~65\%
    \end{align*}
        
    }

    \noindent{3. Draw the wave form of both $u_0$ and $u_1$ when S is set to 2.}
    {\color{hwSolution}
        
        \begin{center}\begin{tikzpicture}
    \draw[hwSolution,thick](-1.2,1.0) node[left,hwSolution]{$u_1$}
        -- ( 0.0,1.0) -- ( 0.0,1.6)
        -- ( 8.0,1.6) -- ( 8.0,1.0)
        -- (12.3,1.0) -- (12.3,1.6)
        -- (13.3,1.6) node[right]{$\dots$};
    \draw[hwSolution,thick](-1.2,0.0) node[left,hwSolution]{$u_2$}
        \foreach \x in {0,...,5}{
            -- ({1.23 * \x},0.0)
            -- ({1.23 * \x},0.6)
            -- ({1.23 * \x + 0.8 },0.6)
            -- ({1.23 * \x + 0.8 },0.0)
        }
        -- ({1.23 * 6},0.0) -- ({1.23 * 6},0.6)
        -- ( 8.0,0.6) -- ( 8.0,0.0)
        -- (12.3,0.0) -- (12.3,0.6)
        -- (13.1,0.6) -- (13.1,0.0)
        -- (13.3,0.0) node[right]{$\dots$};
    \draw[red,dash pattern={on 1pt off 1pt}]( 8.0,0.6)
    -- ({1.23 * 6 + 0.8 },0.6)
    -- ({1.23 * 6 + 0.8 },0.0);
    \draw[black,->,thick] (-1.2,-0.5) -- (14.0,-0.5)
        node[below right]{$t$};
    \foreach \x in {0,...,7}{
        \draw[black,thick] ({1.23 * \x},-0.5)
        -- ({1.23 * \x},-0.35);
    }
    \foreach \x in {0,...,6}{
        \draw[gray,thick,dash pattern={on 2pt off 3pt}] ({1.23 * \x},-0.4)
        -- ({1.23 * \x},1.6);
    }
    \foreach \x in {0,1}{
        \draw[black,thick] ({1.23 * \x + 12.3},-0.5) 
        -- ({1.23 * \x + 12.3},-0.35);
    }
    \draw[gray,thick,dash pattern={on 2pt off 3pt}]
    (12.3,-1.4) -- (12.3, 1.6)
    (0.0,-1.4) -- ( 0.0,-1.15)
    (8.0,1.0) -- (8.0,0.6);
    \usetikzlibrary{snakes}
    \draw[hwSolution,snake = brace,thick] (1.23,-0.6)--(0,-0.6)
    (0.615,-0.7) node[below]{${T_2 = 0.438\,ms}$};
    \draw[hwSolution,snake = brace,thick] (12.3,-1.5)--(0,-1.5)
    (6.15,-1.6) node[below]{${T_1 = 4.38\,ms}$};

\end{tikzpicture}\end{center}

    }