\section{Session 6 - Pulse generating and shaping}
\vspace{-15pt}\noindent\rule{\textwidth}{0.1pt}\vspace{-10pt}
    \subsection*{Session 6 Homework}
    \subsubsection{9.1 \textnormal{ Give the waveform of $u_0$ }.}
    {\color{hwSolution}

        \begin{center}\begin{tikzpicture}
    \draw[black,thick,->] (-0.5,0) -- (10,0) node[below]{$t$};
    \draw[black,thick,->] (0,-0.5) -- ( 0,3.5) node[left]{$u_I$};
    \draw[black,dash pattern={on 0.2cm off 0.2cm}]
    (0,1) node[left]{$U_{T-}$} -- (9.5,1)
    (0,2) node[left]{$U_{T+}$} -- (9.5,2);
    \draw[gray,thick] plot[smooth] coordinates{
    (0,0)
    (0.5,2.0) %N1
    (0.7,2.2)
    (1.1,1.8)
    (1.4,2.1)
    (1.6,1.8)
    (1.9,2.1)
    (2.2,1.0) %N2
    (2.3,0.0)
    (2.8,1.0)
    (3.0,1.3)
    (3.3,1.4)
    (3.5,0.8)
    (3.7,1.2)
    (4.1,0.7)
    (4.6,2.0) %N3
    (5.1,2.3)
    (5.3,1.8)
    (5.5,1.8)
    (5.9,2.6)
    (6.2,1.0) %N4
    (6.6,0.0)
    (7.0,1.3)
    (7.6,0.8)
    (7.9,1.3)
    (8.4,0.8)
    (9.0,1.2)};
    \draw[hwSolution,thick](0,0.5) node[left,hwSolution]{$u_0$}
    -- (0.5,0.5) -- (0.5,2.5)
    -- (2.2,2.5) -- (2.2,0.5)
    -- (4.6,0.5) -- (4.6,2.5)
    -- (6.2,2.5) -- (6.2,0.5)
    -- (9.0,0.5);
\end{tikzpicture}\end{center}
        
    }

    \subsubsection{9.5 \textnormal{Given a 74121 connected as shown}.}
    \noindent{1.Calculate the range of delay time}
    {\color{hwSolution}
        \[
            CR \leq t_d \leq C(R+R_W)~~\\
        \]
        \[
            3.57ms \leq t_d \leq 18.97ms
        \]
    }

    \noindent{2.What's the functionality of the resister next to $R_w$?}
    {\color{hwSolution}
    
        It prevents short circuit when $R_w$ is set to 0.
    
    }

    \subsubsection{9.8}
    \noindent{1. Analyze the status of circuit when S is open.}
    {\color{hwSolution}

        When S remains open, \[\overline{TR}\equiv V_{cc} > \frac{1}{3}V_{cc}\]

        TH will be flipped to Low if it was High, and remains Low as a stable status.

        Hence, $u_0$ holds on $0$. The circuit is stable.

    }

    \noindent{2. Let $C=10\,\mu F$, give the value of $R$ so as the circuit outputs a pulse of $t_w = 10s$ when S is pressed.}
    {\color{hwSolution}

        Since the given design is a standard monostable trigger, we can use $t_w = RC\,ln3 = 10s$.

        Hence, $R = 910\,k\Omega$.

    }

    \noindent{3. What's the value of $R$ if $C=0.1\,\mu F,~~t_w=5\,ms$? What value of $t_w$ do we expect if we replace $C$ by $1\,\mu F$ with the same $R$?}
    {\color{hwSolution}
        \begin{align*}
            t_w = RC\,ln3 &= 5\,ms\\
            R = \dfrac{5\,ms}{0.1\,\mu F\,\cdot\,ln3} &= 45.5\,k\Omega \\
            Replace \Rightarrow& C=1\,\mu F\\
            t_w &= 50\,ms
        \end{align*}
    }

    \subsubsection{9.13 \textnormal{}.}
    \noindent{1. What kind of function dose each of the 555 chip serve?}
    {\color{hwSolution}

        Each of them is a multivibrator.
    }

    \noindent{2. Analyze the status of circuit when S is set to 1.}
    {\color{hwSolution}

        Charging time:
        \begin{align*}
            (Chip1)~~T_{1} &= (R_1+R_2)C\,ln\,2 = 2.84\,ms\\
            (Chip2)~~T_{1} &= (R_1+R_2)C\,ln\,2 = 0.284\,ms
        \end{align*}

        Discharge time:
        \begin{align*}
            (Chip1)~~T_{2} &= R_2\,C\,ln\,2 = 1.53\,ms\\
            (Chip2)~~T_{2} &= R_2\,C\,ln\,2 = 0.153\,ms
        \end{align*}

    Duty cycle:
    \begin{align*}
        (Chip1)~~F &= 228.3\,Hz\\
        (Chip2)~~F &= 2.283\,kHz\\
        Ratio&~65\%
    \end{align*}
        
    }

    \noindent{3. Draw the wave form of both $u_0$ and $u_1$ when S is set to 2.}
    {\color{hwSolution}
        
        \begin{center}\begin{tikzpicture}
    \draw[black,thick,->] (-0.5,0) -- (10,0) node[below]{$t$};
    \draw[black,thick,->] (0,-0.5) -- ( 0,3.5) node[left]{$u_I$};
    \draw[black,dash pattern={on 0.2cm off 0.2cm}]
    (0,1) node[left]{$U_{T-}$} -- (9.5,1)
    (0,2) node[left]{$U_{T+}$} -- (9.5,2);
    \draw[gray,thick] plot[smooth] coordinates{
    (0,0)
    (0.5,2.0) %N1
    (0.7,2.2)
    (1.1,1.8)
    (1.4,2.1)
    (1.6,1.8)
    (1.9,2.1)
    (2.2,1.0) %N2
    (2.3,0.0)
    (2.8,1.0)
    (3.0,1.3)
    (3.3,1.4)
    (3.5,0.8)
    (3.7,1.2)
    (4.1,0.7)
    (4.6,2.0) %N3
    (5.1,2.3)
    (5.3,1.8)
    (5.5,1.8)
    (5.9,2.6)
    (6.2,1.0) %N4
    (6.6,0.0)
    (7.0,1.3)
    (7.6,0.8)
    (7.9,1.3)
    (8.4,0.8)
    (9.0,1.2)};
    \draw[hwSolution,thick](0,0.5) node[left,hwSolution]{$u_0$}
    -- (0.5,0.5) -- (0.5,2.5)
    -- (2.2,2.5) -- (2.2,0.5)
    -- (4.6,0.5) -- (4.6,2.5)
    -- (6.2,2.5) -- (6.2,0.5)
    -- (9.0,0.5);
\end{tikzpicture}\end{center}

    }