\section*{微机原理与接口技术第一次作业}
    \setcounter{section}{1}
    \vspace{-5pt}\subsubsection*{张宇轩 2160909016}
    \vspace{-10pt}\noindent\rule{\textwidth}{0.1pt}
    \vspace{-30pt}
    \subsection{思考:“手机带给我们的”}
        \subsubsection*{社会层面}
            在智能手机面世以前,手机的功能仅限于拨打电话和发送短信,也就是基本通信功能。手机提供的基本通信功能是维持现代社会高效运转的一大基础,它让社会成员的联系从局限的面对面的交流拓展到了随时、随地的联系和沟通。需要注意的是,手机的基本通信功能带来的这一进步对社会的影响比字面上看起来的大得多。如下图所示,它使得社会成员之间的沟通效率大幅提高,从而为人与人之间组织协作的效率带来了数量级上的提升。
            \vspace{20pt}
            \begin{quotation}
                \tikzstyle{unitbox} = [rectangle,rounded corners, minimum width=2cm,minimum height=0.74cm,text centered, draw=black,fill=none]
                \tikzstyle{arrow} = [thick,->,>=stealth]
                \tikzstyle{arrow_ds} = [thick,<->,>=stealth]

                \noindent\begin{tabularx}{\linewidth}{AA}
                    \begin{tikzpicture}[node distance=1.5cm]
                        \node (org) [unitbox] {组织/单位};
                        \node (indv1) [unitbox,below of=org,xshift=-4em] {成员};
                        \node (indv2) [unitbox,below of=org,xshift=+4em] {成员};
                        \draw [arrow_ds] (org) -- (indv1);
                        \draw [arrow_ds] (org) -- (indv2);
                    \end{tikzpicture}
                    \begin{quotation}
                        没有实时通讯工具的情况下,组织成员间必须通过某种中介来互相沟通,或者必须在特定的时间、地点进行沟通。信息无法即时传递,也无法直接传递。
                    \end{quotation}
                    &
                    \begin{tikzpicture}[node distance=1.5cm]
                        \node (org) [unitbox] {组织/单位};
                        \node (indv1) [unitbox,below of=org,xshift=-4em] {成员};
                        \node (indv2) [unitbox,below of=org,xshift=+4em] {成员};
                        \draw [arrow_ds] (org) -- (indv1);
                        \draw [arrow_ds] (org) -- (indv2);
                        \draw [arrow_ds] (indv1) -- (indv2);
                    \end{tikzpicture}
                    \begin{quotation}
                        装备实时通讯工具后,组织成员之间的交流协作不再需要限定时间、地点,在任意时间任意地点都可以与每个取得联系。交流的效率被大大提升了。
                    \end{quotation}
                    \\
                \end{tabularx}
            \end{quotation}
            而移动操作系统和互联网的诞生则为手机增添了另一层面的功能:信息获取和群体社交功能。基础通信功能提供了“点对点”的交流渠道,而基于互联网的新智能平台提供了“点对面”的传播功能和“面到点”的信息获取功能。
            
            这一变化的意义在于,它“拍扁了”传统的社会层级结构,信息与知识的传播不再需要经过一个中心节点(如电视台、报纸、教科书)。每个个体可以直接为社群贡献信息(知识),也可以通过搜索和智能推荐技术方便快速地从社群获取信息(知识)。信息流转的效率和范围都有了巨大的提高。同时,它打破了原有的“熟人社交”模式,将人们按照兴趣爱好、职业特长、年龄段等要素组织起来,创造了新的、基于“群像”的社交模式。
            
            这种高度“扁平化”的高效交流模式和大脑中的神经元如此类似,甚至可以在一定的群体中产生“集体智慧”。

        \subsubsection*{个人层面}
            相比智能设备给社会带来的巨大进步,智能手机对个人的影响就未必那么正面了。由于个体的差异性,这种影响不能一概而论,只能进行基于统计(观察)的抽象概括。

            固然,个体以智能设备作为入口,在互联网世界中获得极大的好处。但是,网络世界信息的获取如此简单,以至于一部分人因此丧失了独立思考、处理问题的能力。同时,大量信息的涌入极有可能导致我们无法专注于一件事情。我们大脑虽然非常强大,但是并不擅长在多个任务之间来回跳跃,就像老式的单线程操作系统。同时抵达手机的大量信息反而会导致我们处理事情的效率大幅降低。
        %\pagebreak

        \subsubsection*{工具属性和娱乐属性}
            您可能已经注意到,我在上面两个部分没有提到手机的工具属性(如计算器)和娱乐属性(游戏、视频播放等)。这是因为,这两部分属性并不是手机所特有的。
            
            手机的工具属性只是对传统工具的集成,虽然提升了这些工具的便携性和易用性,但是并没有对它们的功能做出本质改变。

            而手机的娱乐属性也是一样,手机游戏利用“低成本即时反馈”给玩家带来快乐(刺激)的原理,和电脑游戏、传统棋牌游戏甚至赌博游戏所利用的心理学原理没有本质区别。手机只是作为游戏的载体,并没有对游戏的根本属性做出改变。

            因此,对这部分内容,在此不作讨论。

    \subsection{思考:“兴趣与学习”}
        \subsubsection*{行为与动机}
            在讨论兴趣与学习之前,有必要对行为与动机的联系进行更普适的讨论。通过对于一些心理和逻辑上的问题的思考,我们可以更加深刻地认识兴趣与学习的关系。
            
            首先,任何(理智的人的)行为必然对应一个动机,而动机的本质是获得某种反馈。“反馈”指的是那些可以实在的令人产生(生理或心理的)愉悦感的事件的集合。而这种心理上的对于“反馈”的需求,构成了一个理性人的完整的行为逻辑,即动机。

            由常理不难推断,一件事情(行为)的正面反馈越强,则动机越强,则执行此任务时的专注性和积极性就越强。
        
        \subsubsection*{逻辑链和延迟反馈}
            然而,在当代现实生活中,大部分日常工作并不直接对应一个“反馈”,甚至有些行为还会给我们带来(生理或心理的)负面体验。例如我们每天听课、写作业并不对应一个实际的反馈(奖励),反而会让我们身心疲惫。但是大部分人并不会因此拒绝听课或完成作业,这似乎和前文所述相矛盾。

            为了解决上述问题,我们假定“完成学业”为最终反馈(奖励),那么我们会发现,“完成当日作业”这一行为和最终反馈有如下关系:
            
            \begin{center}
                \tikzstyle{unitbox} = [rectangle,rounded corners, minimum width=2cm,minimum height=0.74cm,text centered, draw=black,fill=none]
                \tikzstyle{caption} = [rectangle,text centered, draw=none,fill=none]
                \tikzstyle{arrow} = [thick,->,>=stealth]
                \tikzstyle{link}  = [thin,<-,>=stealth]
                \begin{tikzpicture}[node distance=3cm]
                    \node (n1) [unitbox] {听课/作业};
                    \node (n2) [unitbox,right of=n1] {获得知识};
                    \node (n3) [unitbox,right of=n2] {通过考试};
                    \node (n4) [unitbox,right of=n3] {获得学分};
                    \node (n5) [unitbox,right of=n4] {完成学业};
                    \draw [arrow] (n1) -- (n2);
                    \draw [arrow] (n2) -- (n3);
                    \draw [arrow] (n3) -- (n4);
                    \draw [arrow] (n4) -- (n5);
                    \node (c1) [caption,below of=n1,yshift=60pt] {行为};
                    \node (c1a)[caption,below of=c1,yshift=70pt] {含有负面体验};
                    \node (c2) [caption,below of=n2,yshift=60pt] {直接结果};
                    \node (c2a)[caption,below of=c2,yshift=70pt] {不产生正面反馈};
                    \node (c5) [caption,below of=n5,yshift=60pt] {最终结果};
                    \node (c2a)[caption,below of=c5,yshift=70pt] {间接作用于行为};
                    \draw [link] (n1) -- (c1);
                    \draw [link] (n2) -- (c2);
                    \draw [link] (n5) -- (c5);

                \end{tikzpicture}
            \end{center}

            任何一个具备基本逻辑思维能力(即理性)的人都能本能地认识到听课、写作业(行为)与完成学业(最终)反馈之间的联系,所以听课、写作业这些表面上不产生有效反馈的行为依旧可以被执行。这种“行为”与反馈间接相关的现象被一部分人称为“延迟反馈”,而某个人执行这类(不产生直接反馈的)任务的能力被称为“延迟反馈能力”。

        \subsubsection*{人为反馈和自我奖励机制}
            上文所述例子已经展现了间接(延迟)反馈的原理,一个理论上完全理性的人应当可以完美遵循以上逻辑。可惜的是,现实中并不存在这种完全理性的人,甚至我们中的大多数人遵循自身理性思维指导的比例都很低。因此,在现实中,最终反馈对具体行为的激励作用是沿逻辑链逐级衰减的,同时也是随时间跨度递减的。
            
            也就是说,一条完整反馈链的逻辑越复杂、时间跨度越长,它的每一步执行起来就越困难。

            \pagebreak

            那么,那些“延迟反馈”能力强的人,是如何克服这种现象的呢?答案就是人为创造直接作用在每一步“具体行为”上的“人为反馈”:
            
            \begin{center}
                \tikzstyle{unitbox} = [rectangle,rounded corners, minimum width=2cm,minimum height=0.74cm,text centered, draw=black,fill=none]
                \tikzstyle{caption} = [rectangle,text centered, draw=none,fill=none]
                \tikzstyle{arrow} = [thick,->,>=stealth]
                \tikzstyle{link}  = [thin,<-,>=stealth]
                \begin{tikzpicture}[node distance=3cm]
                    \node (n1) [unitbox] {听课/作业};
                    \node (n2) [unitbox,right of=n1,yshift=1cm] {获得知识};
                    \node (n3) [unitbox,right of=n2] {通过考试};
                    \node (n4) [unitbox,right of=n3] {获得学分};
                    \node (n5) [unitbox,right of=n4] {完成学业};
                    \draw [arrow] (n1) -- (n2);
                    \draw [arrow] (n2) -- (n3);
                    \draw [arrow] (n3) -- (n4);
                    \draw [arrow] (n4) -- (n5);
                    \node (s2) [unitbox,below of=n2,yshift=1cm] {自我奖励};
                    \draw [arrow] (n1) -- (s2);
                    \node (c1) [caption,below of=n1,yshift=60pt] {行为};
                    \node (c2) [caption,below of=n2,yshift=60pt] {直接结果};
                    \node (c5) [caption,below of=n5,yshift=60pt] {最终结果};
                    \draw [link] (n1) -- (c1);
                    \draw [link] (n2) -- (c2);
                    \draw [link] (n5) -- (c5);
                    \node (cs2) [caption,right of=s2,xshift=20pt] {反馈直接作用于具体行为};
                    \draw [link] (s2) -- (cs2);

                \end{tikzpicture}
            \end{center}

            上图所示的“自我奖励”可以是根据短期计划完成情况安排的休息、用餐,也可以是对自己的某种具体的物质奖励。一种特殊的奖励方式就是“精神上的奖励”,也就是对于完成一项任务或者收获一种知识的成就感。这种成就感可以由个人对一个学科/领域的兴趣产生,也可以由完成计划本身产生,甚至可以是两者的叠加。

            强烈而直接的人为正面反馈会提升完成任务的速度和质量,从而在兴趣和计划性两方面构建起良性循环。

            至此,“兴趣与学习”的问题已经在推论过程中基本阐明,这一推论同时揭示了合理制定计划、构建正面反馈机制对于学习和人生的重大意义。

        \subsubsection*{估计偏差}

            以上反馈模型中用到了一项假设:我们总是可以正确估计一件事的难度(执行过程中的负面体验)和完成一件事的意义(最终结果伴随的正面反馈)。这在现实生活中是不准确的。

            事实上,我们大多数人倾向于低估一件事执行的难度,并夸大一件事最终完成后的正面体验,甚至会在任务完成前尽可能地“提前消费”其正面反馈。这种现象会导致良性的正面反馈循环无法建立,甚至会倒向另一个极端————恶性循环。

            这一问题十分现实,也十分棘手。我个人暂时没有很好的解决办法。

        \subsubsection*{自洽性}

            我在为阐释“兴趣与学习”而建立的这个模型中,通篇没有提及“自制力”在逻辑链中扮演的角色。这不代表本文所建立的模型无法自洽。因为在本模型中,发挥自制力的过程可以被等价为提前消费“最终正面反馈”来抵消当下“负面体验”的行为。它们本质上是等价的。而上文已经提到,提前消费最终正面反馈的行为是不恰当的、危险的。

            有相关心理学实验证明,自制力不是恒定的。一个人使用自制力的过程同时也是消耗自制力的过程。这一结论与本文模型相符。

            换言之,(理想情况下)通过正确构建兴趣体系和计划体系,设置合理的、足够的人为反馈,我们不需要所谓“自制力”参与日常生活。

        \vspace{2cm}

        \hfill \textbf{感谢老师耐心阅读!}
 
        \hfill \textbf{张宇轩}
 
        \hfill \textbf{2160909016}
        
        \hfill \textbf{\today}