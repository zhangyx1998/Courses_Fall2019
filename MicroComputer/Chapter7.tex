\section{Chapter7}
\vspace{-15pt}\noindent\rule{\textwidth}{0.1pt}\vspace{-10pt}
    \subsection*{Chapter 7 Homework}
    \subsubsection{\textnormal{综合设计:}}

    利用8255和8253可编程并行接口设计一个简易的安全报警系统. 功能要求: 

    1. 在房间门窗等8个入口处安装探测器. 正常情况下, 探测器输出为低电平. 当探测到异常时, 探测器输出高电平. 此时, 启动报警(警铃响, 警灯闪烁), 并在危险解除(有任意键按下)后关闭报警;

    2. 当外部向8255的PC0端输入低电平时(可利用开关实现), 监控系统启动. 

    3. 系统启动后, 在初始状态下, 警铃不响, 警灯不亮. 系统不断检测各探测器的输出电平, 如果检测到有任意一个探测器的输出为高电平, 并且在随后的5次连续检测中, 该探测器的输出都为高电平, 则通过8255的PC6启动报警(PC6输出高电平), 使8253通道0产生1kHz频率的方波, 控制警铃发出警报声. 由8255的PC7控制警报灯闪烁(PC7输出高电平时警灯亮). 

设: 8255的端口地址为380H~383H. 8253的端口地址为384H~387H. 外部时钟脉冲2MHz. 

要求: 完成系统硬件设计和软件设计(包括对外部设备的连接示意图)

    \begin{center}\begin{tikzpicture}
    \draw[black,thick,->] (-0.5,0) -- (10,0) node[below]{$t$};
    \draw[black,thick,->] (0,-0.5) -- ( 0,3.5) node[left]{$u_I$};
    \draw[black,dash pattern={on 0.2cm off 0.2cm}]
    (0,1) node[left]{$U_{T-}$} -- (9.5,1)
    (0,2) node[left]{$U_{T+}$} -- (9.5,2);
    \draw[gray,thick] plot[smooth] coordinates{
    (0,0)
    (0.5,2.0) %N1
    (0.7,2.2)
    (1.1,1.8)
    (1.4,2.1)
    (1.6,1.8)
    (1.9,2.1)
    (2.2,1.0) %N2
    (2.3,0.0)
    (2.8,1.0)
    (3.0,1.3)
    (3.3,1.4)
    (3.5,0.8)
    (3.7,1.2)
    (4.1,0.7)
    (4.6,2.0) %N3
    (5.1,2.3)
    (5.3,1.8)
    (5.5,1.8)
    (5.9,2.6)
    (6.2,1.0) %N4
    (6.6,0.0)
    (7.0,1.3)
    (7.6,0.8)
    (7.9,1.3)
    (8.4,0.8)
    (9.0,1.2)};
    \draw[hwSolution,thick](0,0.5) node[left,hwSolution]{$u_0$}
    -- (0.5,0.5) -- (0.5,2.5)
    -- (2.2,2.5) -- (2.2,0.5)
    -- (4.6,0.5) -- (4.6,2.5)
    -- (6.2,2.5) -- (6.2,0.5)
    -- (9.0,0.5);
\end{tikzpicture}\end{center}