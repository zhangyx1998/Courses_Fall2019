\section{Chapter7}
\vspace{-15pt}\noindent\rule{\textwidth}{0.1pt}\vspace{-10pt}
    \subsection*{Chapter 7 Homework}
    \subsubsection{\textnormal{综合设计:}}

    利用8255和8253可编程并行接口设计一个简易的安全报警系统. 功能要求: 

    1. 在房间门窗等8个入口处安装探测器. 正常情况下, 探测器输出为低电平. 当探测到异常时, 探测器输出高电平. 此时, 启动报警(警铃响, 警灯闪烁), 并在危险解除(有任意键按下)后关闭报警;

    2. 当外部向8255的PC0端输入低电平时(可利用开关实现), 监控系统启动. 

    3. 系统启动后, 在初始状态下, 警铃不响, 警灯不亮. 系统不断检测各探测器的输出电平, 如果检测到有任意一个探测器的输出为高电平, 并且在随后的5次连续检测中, 该探测器的输出都为高电平, 则通过8255的PC6启动报警(PC6输出高电平), 使8253通道0产生1kHz频率的方波, 控制警铃发出警报声. 由8255的PC7控制警报灯闪烁(PC7输出高电平时警灯亮). 

设: 8255的端口地址为380H~383H. 8253的端口地址为384H~387H. 外部时钟脉冲2MHz. 

要求: 完成系统硬件设计和软件设计(包括对外部设备的连接示意图)

    \begin{center}\begin{tikzpicture}
    \def \HFH{10}
    \coordinate (DATA) at (5, 0);
    \draw[black]
        \foreach \x in {0,...,7}{
            (DATA)+({\x * 0.2},\HFH)--+({\x * 0.2},-\HFH)
        }
        (DATA)+(0.0,\HFH) node[above,left]{$D_7$}
        (DATA)+(1.4,\HFH) node[above,right]{$D_0$};
    \coordinate (ADDR) at (0, 0);
    \draw[black]
        \foreach \x in {2,...,11}{
            (ADDR)+({\x * 0.1},\HFH)--+({\x * 0.1},-\HFH)
        }
        (ADDR)+(0.0,\HFH) node[above,left]{$A_{11}$}
        (ADDR)+(1.1,\HFH) node[above,right]{$A_0$};
    % 24 Decoder
    \coordinate (C0) at (-4, 0);
    \draw[hwSolution,thick]
        (C0)+(-1,1.6) -- +(1,1.6) -- +(1,-1.6) -- +(-1,-1.6) -- cycle;
    \draw[black]
        (C0)+(-1.0, 0.8) node[right]{$A_0$} -- +(-1.4, 0.8)
        (C0)+(-1.0,-0.8) node[right]{$A_1$} -- +(-1.4,-0.8)
        (C0)+(-1.4, 0.8) -- +(-1.4,{\HFH-1.5}) -- (0.1,{\HFH-1.5}) -- (0.1,\HFH)
        (C0)+(-1.4,-0.8) -- +(-1.5,-0.8) -- +(-1.5,{\HFH-1.4}) -- (0.0,{\HFH-1.4}) -- (0.0,\HFH);
    \draw[hwSolution,thick,o-*] (C0)+( 1.0, 1.2) node[left] {$Y_0$} -- +( 1.4, 1.2) --  (-2.4, 1.2) -- (-2.4, 6.7) -- (-0.9, 6.7);
    \draw[hwSolution,thick,o-*] (C0)+( 1.0, 0.4) node[left] {$Y_1$} -- +( 1.4, 0.4) --  (-2.0, 0.4) -- (-2.0, 1.9) -- (-0.9, 1.9);
    \draw[hwSolution,thick,o-*] (C0)+( 1.0,-0.4) node[left] {$Y_2$} -- +( 1.4,-0.4) --  (-2.0,-0.4) -- (-2.0,-2.9) -- (-0.9,-2.9);
    \draw[hwSolution,thick,o-*] (C0)+( 1.0,-1.2) node[left] {$Y_3$} -- +( 1.4,-1.2) --  (-2.4,-1.2) -- (-2.4,-7.7) -- (-0.9,-7.7);

    \foreach \Chip in {0,...,7}{
        \def \DWXSHIFT {2}
        \ifodd\Chip \def \DWXSHIFT{2.8} \fi
        \coordinate (C\Chip) at (3, {(3.5-\Chip)*2.4});
        \draw[hwSolution,thick]
            (C\Chip)+(-0.8,1) -- +(0.8,1) -- +(0.8,-1) -- +(-0.8,-1) -- cycle;
        \draw[black]
            (C\Chip)+(-0.9,0.8) node(A9C\Chip)[right]{$_{A_9}$}
            (C\Chip)+(-0.9,0.1) node(A0C\Chip)[right]{$_{A_0}$}
            \foreach \ADDRWIRE in {0,...,9}{
                (C\Chip)+(-0.8,{1 - 0.1*(1+\ADDRWIRE)}) -- +({\ADDRWIRE*0.1 - 2.8},{1 - 0.1*(1+\ADDRWIRE)})
            }
            \foreach \DATAWIRE in {0,...,3}{
                (C\Chip)+( 0.8,{0.3-0.2*\DATAWIRE}) node[left,xshift = 3]{$_{_{D_\DATAWIRE}}$} -- +({\DATAWIRE*-0.2 + 0.6 + \DWXSHIFT},{0.3-0.2*\DATAWIRE})
            };
        \draw[black,dotted,thick]
            (A9C\Chip) -- (A0C\Chip);
        \draw[hwSolution,thick,o-]
            (C\Chip)+(-0.8,-0.5) node[right,xshift = -2]{$_{\overline{CS}}$}
            --+(-4,-0.5)
            \ifodd\Chip --+(-4,1.9) \fi;
    }
\end{tikzpicture}\end{center}