\section{Session 1 - The Symbolic System For Group Theory}
\vspace{-15pt}\noindent\rule{\textwidth}{0.1pt}\vspace{-10pt}
    \subsection{Homework}
    \subsubsection{1.12 \textnormal{证明除恒等元外的所有元都是二阶的群是阿贝尔群}.}
    {\color{hwSolution}
    \noindent 证明:

        若:$\forall A_n \in G ,~A_n \neq E ,~A_n^2 = E$

        且由群的封闭性可知:$\forall A_j,\,A_k \in G,\,A_j\,A_k \in G$
        
        即 $A_j\,A_k$ 也满足\,$\left(A_j\,A_k\right)^2 = E$

        所以:
        \begin{align*}
            A_j\,A_k    &= A_j\,\cdot\,E\cdot\,A_k\\
                        &= A_j\,\left(A_j\,A_k\,A_j\,A_k\right)\,A_k\\
                        &= \left(A_j\,A_j\right)\,A_k\,A_j\,\left(A_k\,A_k\right)\\
                        &= A_k\,A_j
        \end{align*}

        即 $A_j\,A_k = A_k\,A_j~(j\neq k)$ 对群$G$内的所有群元成立.

        所以,原命题得证.
    }

    \subsubsection{1.14 \textnormal{证明每个(所有)指数为2的子群是正规子群}.}
    {\color{hwSolution}
    \noindent 证明:

        设群$G$及其子群$S$满足 $g = 2s$,并设$X\,\in\,G$,则:

        (1) 若$X\,\in\,S$, 则显然:
        \[ XS\,=\,SX\,=\,S \]

        (2) 若$X\,\notin\,S$,则:
        \[\forall S_i \in S,\,XS_i\,\notin\,S,\,XS_i\,\in\,G\]
        \[\forall S_i,\,S_j \in S\,(i\neq j),\,XS_i\,\neq\,XS_j\]

        相似地:
        \[\forall S_i \in S,\,S_iX\,\notin\,S,\,S_iX\,\in\,G\]
        \[\forall S_i,\,S_j \in S\,(i\neq j),\,S_iX\,\neq\,S_jX\]

        即:
        
        集合 $XS,\,SX$ 均含有 $s$ 个各不相同, 且不属于 $S$ , 但属于 $G$的元,
        
        又因为 $G$ 中不属于 $S$ 的元的数量与 $S$ 的元数量相等(均为 $s$ 个),所以:
        \[ XS\,=\,SX\,=\,G-S \]

        综合(1)~(2)两种情况: 
        \[~\forall X\,\in\,G,\,XS\,=\,SX~\]

        所以,原命题得证.
    }

    \subsubsection{1.15 \textnormal{若 $G = H \otimes K$ 证明:}}
    \noindent (1)商群 $ G / H $ 与 $ K $ 同构;

    {\color{hwSolution}
    \noindent 证明:
        
        直积群可以展开为:
        \begin{align*}
            G = H \otimes K
                &= \sum\limits_{i,j}^{} H_j K_i\\
                &= \sum\limits_{i}^{} H\cdot K_i
        \end{align*}

        由商群的定义可知:
        \[ G/H\,=\,\{HA_1,HA_2,...,HA_k\} \]

        其中, $k$ 是群 $K$ 的阶, $A_1$ 是群 $H,K,G$ 的共同单位元 $E$.

        不失一般性,假设 $A_n = H_j K_i\,(n\neq 1)$:
        \begin{align*}
            H A_n
                &= H H_j K_i \\
                &= (H H_j) K_i \\
                &= H K_i
        \end{align*}

        即:  $\{A_2,...,A_k\}$ 与 $\{K_2,...,K_k\}$ 一一对应

        比较直积群的展开和商群的定义,加上上述对应关系,可以得出:
        \[ G/H\,=\,\{HK_1,HK_2,...,HK_k\} \]
        
        构造同构对应关系:
        \[ HK_n \leftrightarrow K_n \]

        利用直积的对易关系($K_i H = H K_i$),验证同构性质:
        \begin{align*}
            (H K_i) (H K_j) = H^2 K_i K_j &= H (K_i K_j) \\
            K_i K_j &= (K_i K_j) 
        \end{align*}

        所以,原命题得证.
    }

    \noindent (2) $ G $ 与 $ H $ 及 $ K $ 同态;

    {\color{hwSolution}
    \noindent 证明:
        
        对于 $G$ 与 $H$, 构造同态关系:
        \[ H_i K_j \rightarrow H_i\,(1\leq i\leq h,\,1\leq j\leq k)\]

        利用直积的对易关系,验证同态性质:
        \begin{align*}
            (H_i K_n) (H_j K_n) &= (H_i H_j) K_n^2 \\
            H_i H_j &= (H_i H_j)
        \end{align*}

        由于 $K_n^2 = K_{n'} \in K$, 所以映射 $(H_i H_j) K_n^2 \rightarrow (H_i H_j)$ 依然成立.

        $G$ 与 $K$ 的同态映射构造及证明与上述过程完全一致,不再赘述.

        所以,原命题得证.
    }

    \subsubsection{1.18 \textnormal{证明二阶循环群与四阶循环群同态}.}
    {\color{hwSolution}
    \noindent 证明:

        不失一般性,令
        \begin{align*}
            A&=\{E,\alpha\}&(E=\alpha^2)\\
            B&=\{E,\beta,\beta^2,\beta^3\}&(E=\beta^4)
        \end{align*}

        则可以构造映射关系:
        \begin{align*}
            B_1,\,B_3 \rightarrow A_1\\
            B_2,\,B_4 \rightarrow A_2
        \end{align*}

        验证同态性质:
        \begin{align*}
            \left.\begin{array}{rl}
                B_1\,B_1 &= B_1\\
                B_1\,B_3 &= B_3\\
                B_3\,B_1 &= B_3\\
                B_3\,B_3 &= B_1
            \end{array}\right\}
            \rightarrow A_1A_1 = A_1 &~~~~&
            \left.\begin{array}{rl}
                B_1\,B_2 &= B_2\\
                B_1\,B_4 &= B_4\\
                B_3\,B_2 &= B_4\\
                B_3\,B_4 &= B_2
            \end{array}\right\}
            \rightarrow A_1A_2 = A_2\\
            \left.\begin{array}{rl}
                B_2\,B_1 &= B_2\\
                B_2\,B_3 &= B_2\\
                B_4\,B_1 &= B_4\\
                B_4\,B_3 &= B_4
            \end{array}\right\}
            \rightarrow A_2A_1 = A_2 &~~~~&
            \left.\begin{array}{rl}
                B_2\,B_2 &= B_1\\
                B_2\,B_4 &= B_1\\
                B_4\,B_2 &= B_3\\
                B_4\,B_4 &= B_3
            \end{array}\right\}
            \rightarrow A_2A_2 = A_1
        \end{align*}

        所以,原命题得证.
    }

    \subsubsection{1.19 \textnormal{在有限群中有一组元的集合 $ S $, 对于群乘是封闭的, 试证明集合 $ S $ 中必包含单位元及各元的逆元}.}
    {\color{hwSolution}
    \noindent 证明:

        用 $S = \{S_1,S_2,...,S_s\}$ 表示群 $S$.

        已知: $\forall S_n,S_m\in S,\,\,S_n S_m = S_k\in S$.
    
        且由于S是一有限群的子集,所以S的群元依然满足结合律,并且S中的群元各不相同.\\

        (1)首先证明单位元存在.
        
        由群乘的封闭性, $\exists S_k \in S$, 使得:
        \[ S_1\cdot S_2\cdot...\cdot S_s = S_k \]

        即:
        \[ (S_1\cdot S_2\cdot...\cdot S_{k-1}\cdot S_{k+1}\cdot...\cdot S_s) S_k = S_k \]

        再次利用群乘的封闭性, 可知 $\exists S_t \in S,\, (S_t\neq S_k)$, 使得:
        \[S_t = (S_1\cdot S_2\cdot...\cdot S_{k-1}\cdot S_{k+1}\cdot...\cdot S_s)\]

        故上式简化为:
        \[ S_m\cdot S_k = S_k \]

        所以:
        \[ S_m = E~~or~~S_k = E \]

        上述步骤完成了群 $S$ 中存在单位元的证明.\\

        (2)接下来证明S中包含各元的逆元.
        
        不失一般性, 我们设 $S_1 = E$.
        
        再次利用群乘的封闭性, $\exists S_k \in S$, 使得:
        \[ S_2\cdot S_3\cdot...\cdot S_s = S_k \]

        1.~如果 $S_k = E$,那么由于 $S$ 中群元各不相同,可知 $\forall S_t \in S,\, (S_t\neq S_k)$, 使得:
        \[S_t\cdot(S_2\cdot S_2\cdot...\cdot S_{t-1}\cdot S_{t+1}\cdot...\cdot S_s) = E\]
        \[S_1\cdot S_2\cdot...\cdot S_{t-1}\cdot S_{t+1}\cdot...\cdot S_s= S_t^{-1} \neq S_t\]

        2.~如果 $S_k \neq E$,那么:
        \[S_k\cdot(S_2\cdot S_2\cdot...\cdot S_{k-1}\cdot S_{k+1}\cdot...\cdot S_s) = E\]
        \[S_1\cdot S_2\cdot...\cdot S_{k-1}\cdot S_{k+1}\cdot...\cdot S_s= S_k^{-1} \neq S_k\]

        将 $S_k$ 从上述连乘序列中剔除, 我们就得到第一种情况,从而证明了任意的群元都有对应的逆元.\\

        综上,原命题得证.
    }