\section{The opening}
\vspace{-15pt}\noindent\rule{\textwidth}{0.1pt}\vspace{-10pt}
    \subsection{\hfill \small Friday Sep 6}
    
        \paragraph{Defination of Group}
        \subparagraph{一般性群运算}集合 $G = \{g_1,g_2,...,g_\alpha,...\}$ ,在 $G$ 中定义乘法运算: $G*G\rightarrow G$ ,如果G中元素在这种定义下满足:
        %\begin{itemize}
        %    \item[交换律]
        %\end{itemize}


        \subparagraph*{例:空间反演群}对三位实向量$\vec{r}$构成的群$R^3$ ,空间反演操作 $I$ 定义为 $I\cdot \vec{r}=-\vec{r}$ ,那么 $I^2=E$, 即 $G=\{E,I\}$ 是一个群。

        \subparagraph*{群的阶}
        \subparagraph*{群按阶的分类}
        \subparagraph*{循环群}
        \subparagraph*{乘法表}
        \begin{center}
            \begin{tabular}{c|c c}
                ~ & E & I \\
                \hline
                E & E & I \\
                I & I & E \\
            \end{tabular}
        \end{center}
        
        \subparagraph*{例:n阶置换群}将$n$个元素的集合 $X=\{1,2,...,n\}$映射为自身的置换为:
        \begin{equation*}
            P=
            \left(
                \begin{array}{cccc}
                    1   & 2     & ... & n   \\
                    m_1 & m_2   & ... & m_n \\
                \end{array}
            \right)
        \end{equation*}
        \begin{center}
            \bfseries{注意区分群置换和矩阵变换}
        \end{center}
        \subparagraph*{群可以按交换性分类}满足交换性的群也被称为Abel群。
        \subparagraph*{例:二面体群$D_3$乘法表}如下图所示,正三角形$ABC$,定义如下操作:
        \noindent\begin{tabularx}{\linewidth}{AA}
            abc
                &
            \begi{itemize}
                \item[a]绕体心$O$在$\vec{z}$方向旋转$\frac{2k\pi}{3}\textit{rad}(k\in N)$。
                \item[b]绕X在$\vec{z}$方向旋转$\frac{2k\pi}{3}\textit{rad}(k\in N)$。
            \end{itemize}
            \\
        \end{tabularx}
        \subparagraph*{生成元}
        \subparagraph*{生成关系}
        
    \subsection*{Homework}
        \subsubsection*{[abstract]}
        \subsubsection{[Thinking] To be added}