\section{Week1 Notes\hfill\normalsize\textit{Get into the quantum world}}\SectionRule
    \subsection{Notes on Lesson 1}
    \subsection{Notes on Lesson 2}
    \subsubsection{Bohr-Sommerfeld Quantization (Old quantum theory)}
    \[
        \oint p \,dq = nh
    \]
    While $p$ is the momentum of a particle, which is a function of position of particle $q$. This equation constrains the movement of a particle in a specific potential field V(q).
    \subsection{Notes on Lesson 3}
    \subsubsection{Born Statistical Interpretation of Wave Function}
    The wave function $\Psi (x,t)$ discribes the "Amplitude" of the probability finding the paticle at coordinate(x,t). While the probability density equals amplitude squared.

    That is, the probability finding the particle among $x\in [a,b]$ at time $t_0$ is discribed as:
    \begin{equation*}
        P(x\in [a,b],t=t_0)=\int^{b}_{a}|\psi|^{2}dx=\int^{b}_{a}\psi^*\psi dx
    \end{equation*}

    Or, to say, the \textit{Probability Density} can be written as:
    \begin{equation*}
        p(x,t)=|\psi(x,t)|^2=\psi^{*}(x,t)\psi(x,t)
    \end{equation*}