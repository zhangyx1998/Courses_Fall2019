\section{To be edited}\SectionRule
    \subsection{\hfill \small Measurements and Operators - Monday Sep 16}
    \subsection*{Homework}
    \subsubsection*{[week3 hw1]}
    \subsubsection{Griffiths 2.1}
    \noindent{
        Prove the following three theorems:
    }
    
    \noindent{
        (a) For normalizable solutions, the separation constant $E$ muust be $real$.
    }

    {\color{hwSolution}
        Proof:
    }
    
    \noindent{
        (b) The time-independent wave function $\psi(x)$ can always be taken to be $real$ (unlike $\Psi(x,t)$, which is necessarily complex). This doesm't mean that every solution to the time-independent Schrödinger equation $is$ real; what it says is that if you've got one that is $not$, it can always be expressed as a linear combination of solutions (with the same energy) that $are$. So you $might~as~well$ stick to $\psi'$s that are real.
    }

    {\color{hwSolution}
        Proof:
    }
    
    \noindent{
        (c) If $V(x)$ is an \textbf{even function} (that is, $V(-x)=V(x)$) then $\psi(x)$ can always be taken to be either even or odd.
    }

    {\color{hwSolution}
        Proof:
    }
    \subsubsection{Griffiths 2.2}
    \noindent{
        Show that $E$ must exceed the minimum value of $V(x)$, fot every normalizable solution to the time-independent Schrödinger equation. What is the classical analog to this statement? \textit{Hint:} Rewrite Equation 2.5 in the form
        \[
            \dfrac{d^2\psi}{dx^2}=\dfrac{2m}{\hbar}\left[V(x)-E\right]\psi
        \]
        if $E\leq V_min$, then $\psi$ and its second derivative always have the \textit{same sign} -- argue that such a function cannot be normalized.
    }
    {\color{hwSolution}
        
    }

    \subsubsection{Rewrite the entanglement of XXX when $c_n$ and $\psi_n$ are not necessarily $real$.}