\section{Get to know the quntum world}
\vspace{-15pt}\noindent\rule{\textwidth}{0.1pt}\vspace{-10pt}
    \subsection{\hfill \small Monday Sep 2}
    \subsection{\hfill\small Tuesday Sep 3}
    \subsection{\hfill \small Friday Sep 6}
    \subsubsection*{Born Statistical Interpretation of Wave Function}
    The wave function $\Psi (x,t)$ discribes the "Amplitude" of the probability finding the paticle at coordinate(x,t). While the probability density equals amplitude squared.

    That is, the probability finding the particle among $x\in [a,b]$ at time $t_0$ is discribed as:
    \begin{equation*}
        P(x\in [a,b],t=t_0)=\int^{b}_{a}|\psi|^{2}dx=\int^{b}_{a}\psi^*\psi dx
    \end{equation*}

    Or, to say, the \textit{Probability Density} can be written as:
    \begin{equation*}
        p(x,t)=|\psi(x,t)|^2=\psi^{*}(x,t)\psi(x,t)
    \end{equation*}
    \subsection*{Homework}
        \subsubsection*{[week1 hw1]}
        \subsubsection{[Thinking][Optional] A Brief History of Clues}
        \subsubsection{Caculate wave length of an electromn after a 1000V potential field}
        \subsubsection{Derivation of Klein-Golden equation}
        \subsubsection*{[week1 hw2]}
        \subsubsection{Griffiths 1.2}
        \noindent{
            (a) Find the standard deviation of the distribution in Example 1.1.
        }

        \noindent{
            (b) What is the probability tha a photograph, selected at ramdom, would show a distance $x$ more than one standard deviation away from tha average?
        }
        \subsubsection{Griffiths 1.4}
        \noindent{
            At time $t=0$ a particle is represented by the wave function
            \begin{equation*}
                \Psi(x,0)=
                \left\{\begin{array}{ll}
                    A\dfrac{x}{a}, & if~~0\leq x\leq a, \\
                    \\
                    A\dfrac{(b-x)}{(b-a)}, & if~~a\leq x\leq b, \\
                    \\
                    0, & otherwise,
                \end{array}
                \right.
            \end{equation*}
            where $A,a$ and $b$ are constants.
        }
        
        \noindent{
            (a) Normalize $\Psi$ (that is, find A in terms of $a$ and $b$)
        }

        \noindent{
            (b) Sketch $\Psi (x,0)$, as a function of $x$.
        }

        \noindent{
            (c) Where is the particle most likely to be found, at $t=0$?.
        }

        \noindent{
            (d) What is the probability of finding the particle to the left of $a$? Check your result in the limiting cases $b=a$ and $b=2a$.
        }

        \noindent{
            (e) What is the expectation value of $x$?
        }